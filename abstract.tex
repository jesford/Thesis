%% The following is a directive for TeXShop to indicate the main file
%%!TEX root = diss.tex

\chapter{Abstract}
%Limit of 350 words; is currently (still) at 349!

Clusters of galaxies offer a unique window for studying the Universe on the largest scales. As the most massive gravitationally bound systems to have formed, they serve as probes of the large-scale distributions of dark matter, the underlying cosmology, and the complicated intracluster physics that characterizes the evolution of these massive systems. Gravitational lensing is the deflection of light coming from distant sources, by gravitational potentials along its path. Being sensitive to all mass regardless of type or dynamical state, lensing is a valuable tool for studying dark matter and characterizing galaxy clusters. In the weak lensing regime, the very slight apparent distortion of galaxy shapes is referred to as the shear, while the focusing and amplification of light is referred to as the magnification. The former has become a well-developed and robust technique in astronomy over the past decade, but the latter has been largely overlooked until now.

The work embodied in this thesis includes the first-ever significant detection of magnification by galaxy groups, and the first comparison between masses measured with weak lensing magnification and shear (Chapter 2). This is followed by an application to an enormous sample of galaxy clusters, yielding ground-breaking signal-to-noise for magnification and an analysis of redshift-dependent systematic effects. This project also provides measurements of the cluster mass-richness scaling relation, and is a milestone in moving from magnification detection to useful science (Chapter 3). Finally, a comprehensive gravitational lensing shear analysis is performed on the previous cluster sample, allowing for a critical comparison between cluster masses measured with the independent techniques, as a function of both richness and redshift. These shear measurements also allow for important constraints on a new sample of galaxy clusters, including the distribution of cluster centroid offsets, the mass-richness relation, and cluster redshift evolution (Chapter 4). 

This thesis details unprecedented measurements using a new technique -- weak lensing magnification -- and comparisons with the well-studied shear approach. The final product exemplifies the promise of the new method for measuring galaxy cluster masses, and also points to likely issues that will need to be addressed in future experiments.



%This thesis explores and develops the measurement and interpretation of weak lensing magnification, as applied to galaxy cluster gravitational lenses. The focus is on a particular approach to measuring magnification -- the effect that flux amplification has on measured background galaxy number counts, in a flux-limited survey. This approach is desirable because it can make use of unresolved background sources, and is the most promising avenue for pushing lensing investigations to higher-redshift.




% Consider placing version information if you circulate multiple drafts
%\vfill
%\begin{center}
%\begin{sf}
%\fbox{Departmental Defense Version, Dated: \today} \\
%%\fbox{Draft: \today}
%\end{sf}
%\end{center}
