%% The following is a directive for TeXShop to indicate the main file
%%!TEX root = diss.tex

\chapter{Conclusions}
\label{ch:conc}


%%%%%%%%%%%%%%%%%%%%%%%%%%%%%%%%%%%%%%%%%%%%%%%%%%%%%%%%%%%%%%%%%%%%%%
\section{Thesis Summary}
\label{sec:summary}

This thesis has presented novel results in the study of galaxy clusters through the use of weak gravitational lensing techniques. The scientific domain is the field of cosmology, which was discussed in \autoref{sec:Cosmology}. Cosmologists seek to understand the nature of the Universe on the largest scales -- its history, evolution, and fate, its large scale structure, and all the components of its energy density. This therefore requires knowledge of the matter distribution in the Universe, and how that has changed over time. 

Matter makes up a significance portion of the energy density of the Universe ($\sim30$\% today), and almost all of it is composed of dark matter, which appears to interact only through the gravitational force (see \autoref{sec:DM}). Matter was the dominant component of the Universe for much of its history -- from the time of matter-radiation equality, when the scale factor was just $a(t) \approx 2.9 \times 10^{-4}$, until the recent transition to the current dark energy domination \citep{PlanckXIII_15}. A very useful way to probe the dark matter distribution is through the study of gravitationally collapsed halos, such as those that house galaxies and galaxy clusters.

The specific importance and usefulness of galaxy cluster studies was discussed in \autoref{sec:Clusters}. The main goal, regarding galaxy clusters in this thesis, was to improve measurements of clusters masses, which are so important for both cosmological and astrophysical purposes. The methods employed in this thesis for improving galaxy cluster mass estimates consisted of two different and complementary approaches to weak gravitational lensing. Weak lensing is the subfield of gravitational lensing (see \autoref{sec:Lensing}), wherein light from background sources is subtly bent by gravitational potentials along its path, leading to the focusing and distortion of background galaxies. Unlike strong lensing, where lensing-induced features like giant arcs and multiple images are usually visible to the eye in images, weak lensing is not strong enough to produce obvious effects, but instead is used in a statistical sense, by averaging over many galaxies.

The first of the two weak lensing techniques, weak lensing shear, was explained in \autoref{sec:Shear}. Shear is ubiquitous in the weak lensing literature, and involves measuring slight distortions in image shapes. The second technique, magnification, was explained in \autoref{sec:Mag} and is the dominant focus of the thesis. Magnification is a relatively underused technique, compared to weak lensing shear, and its complementarity with the latter provided the motivation for this thesis research. Both types of weak lensing are statistical in nature, only measurable using many thousands of background galaxies behind many galaxy clusters or other gravitational lenses.  Magnification can be measured in several different ways, but the approach explored in this thesis uses the lensing-induced modifications to the source number densities that are detectable behind a gravitational lens.

The body of this thesis was composed of three published articles, sandwiched between an introduction to the relevant topics in \autoref{ch:Introduction}, and the current concluding chapter. These three peer-reviewed journal articles were all scientifically-led by the thesis author. The publication details can be found either in the Thesis Preface or in the Bibliography references for \citet{Ford12}, \citet{Ford14}, and \citet{Ford15}. 

\autoref{ch2} was comprised of the first magnification study applied to galaxy groups. Prior magnification studies were of galaxy lenses or of very massive (i.e. strongly lensing) clusters. This study yielded the first test of the technique on intermediate mass scales (several times $10^{13}\ {\rm M}_{\odot}$), and also included the first direct comparison between shear-measured masses and magnification-measured masses. These mass estimates were found to be in agreement, albeit with rather large uncertainties. The signal-to-noise from each technique was quantified and compared -- magnification yielded a 4.8$\sigma$ detection, whereas shear achieved 11$\sigma$.

In Chapters \ref{ch3} and \ref{ch4}, a much larger galaxy cluster sample was explored -- the \ac{3D-MF} cluster catalog from the \ac{CFHTLS}-Wide fields. This is one of the largest galaxy cluster catalogs that has been compiled (with over 18,000 clusters), notable containing many lower mass galaxy groups, and covering a wide range of redshifts, up to $z \sim 1$. Optimized with the goal of finding as many clusters as possible, the \ac{3D-MF} cluster finder produced a catalog that is 100\% complete for clusters with mass greater than $3 \times 10^{14}\ {\rm M}_{\odot}$ and 88\% complete above $10^{14}\ {\rm M}_{\odot}$ \citep[see \autoref{sec:3DMF4} or the original \ac{3D-MF} paper][for more details]{Milkeraitis10}. Compared with the small sample of X-ray selected galaxy groups analyzed in \autoref{ch2}, the \ac{3D-MF} clusters in \autoref{ch3} and \autoref{ch4} allow for superior signal-to-noise, even when the clusters are binned according to different attributes. This allows for important studies of clusters as a function of redshift and cluster richness (number of galaxies).

\autoref{ch3} focuses on the magnification signal of the \ac{3D-MF} clusters, which was detected at a significance of $9.7\sigma$. This chapter described the first magnification analysis with a large enough cluster sample to be able to bin as a function of richness \citep[however, note that the publication of this work was followed in quick succession by another magnification study of clusters and luminous red galaxies in the \ac{SDSS} by][]{Bauer14}. Using the richness-binned cluster mass estimates, a power-law scaling relation between cluster mass and richness was determined. This work compared the goodness of fit for two different types of cluster miscentering models -- the model assuming that \ac{3D-MF}'s selected cluster centers were perfectly accurate did not always fit the data as well as the model that assumed an offset distribution, with offsets based on simulations.

Additionally, the masses of clusters binned as a function of redshift were obtained, yielding unexpected variation of mass across redshift bins that had very similar richness distributions. This finding inspired further efforts to model a potential systematic effect for magnification -- physical overlap between lenses and sources in redshift space. Assuming some fraction of source contamination, the expected physical clustering signal was calculated and included in the models that were fit to the magnification signal. This was the first time such an approach had been attempted -- previous magnification studies had chosen to simply avoid interpreting any measurements in redshift regions where contamination is expected to be significant. Judgment regarding the success of this approach was withheld until the follow-up shear study of \autoref{ch4} was completed, in order to follow a semi-blind approach and avoid applying confirmation bias to the magnification results.

\autoref{ch4} paralleled the previous chapter in many respects. This included a shear analysis in the same spirit as the magnification one in \autoref{ch3}, using the publicly available shear measurements from \ac{CFHTLenS}. The binning of clusters was identical in each analysis, in order to produce a side-by-side comparison of the results from each of the two techniques. At 54$\sigma$, the shear signal measured from the cluster lenses is very strong, and this study was able to go beyond what was possible with magnification. In addition to measuring the mass-richness scaling relation, this analysis also searched for any evidence of this relation's evolution with cluster redshift. No significant evolution was detected, which is in line with other recent studies \citep{Andreon14}. 

This work also placed constraints on the \ac{3D-MF} cluster centroid offsets, which was an improvement over the magnification analysis of \autoref{ch3}, and was possible because shear is more sensitive to offset halos. Instead of comparing a perfectly centered model with an offset model, the shear analysis actually fit for the distribution of offsets, finding it to be reasonably well described by a Gaussian, peaking at a radial offset of $\sigma_{\rm off} \sim 0.4$ arcmin (although see \autoref{richbintable14} and \autoref{ztable14} for exact offset values, as well as other parameters). 

A thorough comparison of the results obtained for the \ac{3D-MF} clusters with other cluster analyses in the literature was presented in \autoref{sec:disc4}. Briefly, the slope of the mass-richness relation determined in this thesis was in line with other similar work \citep{Wen12,Covone14}, while the normalization is less easily compared, because it is highly sensitive to the definition of richness used for the sample (which varies widely in the literature). The cluster miscentering offsets measured in \autoref{ch4} of this thesis were intermediate between other results in the literature, which ranged from about half the radial offsets of \ac{3D-MF} \citep{George12} to several times larger than \ac{3D-MF} \citep{Johnston07}.

The research in this thesis pushed the limits of maximizing the extraction of weak lensing information to learn about cluster dark matter halos in our Universe. Two major analyses incorporating magnification information were completed, with some success, but also generated some doubt about the reliability of magnification when redshift contamination of sources is not well known. Prospects for improving the systematics-handling of future magnification studies will be discussed in \autoref{sec:future}. Regardless of any shortcomings of the technique, the undertaking of the detailed magnification studies in this thesis has added value to the weak lensing literature. This body of work provides a framework for accounting for systematic effects in magnification, and highlights issues and important considerations for future studies to build upon.


%%%%%%%%%%%%%%%%%%%%%%%%%%%%%%%%%%%%%%%%%%%%%%%%%%%%%%%%%%%%%%%%%%%%%%
\section{Final Conclusions}
\label{sec:conc}

The overall important findings produced by this thesis research can be summarized as a list of key take-away messages.
\begin{itemize}
\item Magnification can be successfully measured for galaxy groups and clusters, achieving up to half the signal-to-noise as the more commonly-measured shear technique (Chapters \ref{ch2}, \ref{ch3}, and \ref{ch4}).
\item It is possible to obtain consistent cluster mass estimates using weak lensing magnification and shear, at least when low-redshift contamination of sources (for magnification) is minimal (Chapters \ref{ch2} and \ref{ch4}).
\item The \ac{3D-MF} galaxy cluster catalog is publicly available as a result of this work (\autoref{ch4}).
\item The \ac{3D-MF} galaxy clusters exhibit a robust scaling between mass and richness (Chapters \ref{ch3} and \ref{ch4}).
\item The mass-richness scaling relation does not evolve significantly with redshift (\autoref{ch4}).
\item The cluster detection significance of the \ac{3D-MF} clusters scales with mass and can be used as an alternative mass proxy (\autoref{ch4}).
\item The miscentering of the \ac{3D-MF} clusters is significant (though less severe than for some other cluster catalogs) and must be accounted for to avoid biased lensing mass estimates (Chapters \ref{ch3} and \ref{ch4}).
\item The effects of cluster miscentering are degenerate with cluster concentration (\autoref{ch4}).
\item It is confirmed that magnification is highly sensitive to source contamination. Source redshifts and contamination fractions must be known very accurately, and underestimation may have devastating effects on recovered magnification-based masses (Chapters \ref{ch3} and \ref{ch4}).
\item Additional unconsidered systematic effects may plague magnification as well. Specific redshift intervals yielded anomalous magnification mass estimates that were difficult to blame on source contamination (\autoref{ch4}).
\end{itemize}

%%%%%%%%%%%%%%%%%%%%%%%%%%%%%%%%%%%%%%%%%%%%%%%%%%%%%%%%%%%%%%%%%%%%%%
\section{Future Prospects}
\label{sec:future}

There are a number of avenues of future research that could build upon the work in this thesis. The possibilities can be broken roughly into three major areas, which will each be discussed below. They include: galaxy cluster catalog comparisons for different cluster-finding techniques; galaxy cluster halo characterization and study; and the future of weak lensing magnification measurements in the era of upcoming large surveys. One common endeavor that will increase the progress of scientific results in general, will be encouragement of open science through making software and data products public and accessible for others to build upon.

The \ac{3D-MF} cluster catalog has many unique and important characteristics, and future work should compare the cluster sample to catalogs compiled using different cluster-finding techniques. For example, \ac{3D-MF} does not use any color information (aside from photometric redshifts) nor does it require a galaxy to be colocated with the cluster center. Many other cluster-finders rely on the red-sequence of member galaxies and assume that a luminous red galaxy lies at the center. This may allow \ac{3D-MF} to pick up less massive or evolved clusters, similar to our own local group. The careful comparison of different catalogs would help quantify real differences between the cluster samples recovered, and importantly would assist in removing false detections from the \ac{3D-MF} catalog (which are expected to be a significant fraction of the low-mass cluster candidates). Running the \ac{3D-MF} cluster-finder on the same optical data set, alongside other cluster-finders, and analyzing the different objects recovered, would also illuminate the overlap and complementarity of different techniques.

The issue of halo miscentering is interesting for a couple of reasons: (1) weak lensing mass estimates will be biased if miscentering is significant and not accounted for in modeling the lensing profile; (2) clusters that are poorly centered may represent a population of newly forming clusters, some of which might be undergoing mergers, and quantifying their presence and characteristics could be a useful probe of cluster and large-scale structure evolution. Future research should investigate the nature of the offset clusters in the \ac{3D-MF} catalog, to discover whether their centers are merely misidentified, or whether interesting cluster morphology exists. Additionally, alternative center definitions should be explored for the \ac{3D-MF} clusters. Side-by-side shear profile comparisons could demonstrate better center finding for some or all of the clusters, over the original definition employed by the \ac{3D-MF} algorithm.

More broadly, all weak lensing cluster studies need to consider the effects of miscentering. Justification should be given if the distribution of offsets is not accounted for in a weak lensing shear analysis. Studies that purport to measure cluster concentration should be careful to rule out miscentering, the effect of which mimics a low-concentration dark matter halo, by reducing the amplitude of the shear profile at small radii. 

Future magnification studies will need to carefully address the serious systemic effect of the contamination of background sources with objects that overlap with the low-redshift lens population. This can be approached either by improving redshift estimates for sources, so that a pure background sample can be obtained, or by simply avoiding regions of redshift overlap. Additional possible sources of systematics need to be quantified, and some work is being done in this regard by Hildebrandt et al. (private communication). Possible issues may include variations across the images in depth, atmospheric seeing, and contamination by the light halos of stars.

Within the next decade several important large astronomical surveys will begin collecting data. The \acf{LSST} is an 8.4-m telescope, currently under construction in Chile. Starting around 2021, \ac{LSST} will spend 10 years surveying 30,000 deg$^2$ of sky (nearly 200 times the area of \ac{CFHTLenS}) in 6 optical to infrared wavelength bands, providing unprecedented wide field data for weak lensing studies and many other astronomical pursuits \citep{LSST2.0}. \acs{Euclid} is a European Space Agency 1.2-m space-based telescope, planned for launch in 2020, with the goal of improving dark energy constraints. It will spend 6 years obtaining deep imaging of 15,000 deg$^2$ of sky, and weak gravitational lensing is one of the central focuses \citep{Euclid}. The \acf{WFIRST} is a NASA 2.4-m space-based telescope, planned for launch by 2024. Among its other scientific goals, \ac{WFIRST} will perform a weak lensing survey of 2,200 deg$^2$, employing six filters in the near-infrared wavelength range \citep{WFIRST}. \textcolor{red}{Add something about lensing at different z to constrain DE.} 

Weak gravitational lensing is a common central focus of all major upcoming missions and surveys, as it has become an indispensable tool for probing the dark matter, geometry, and growth of structure in our Universe. Magnification, despite some limitations discussed in this thesis, offers great benefits and additional opportunities for these surveys. Especially for ground-based surveys like \ac{LSST}, which will be affected by atmospheric seeing, the ability to use unresolved sources increases the number of gravitationally lensed galaxies that can be included in an analysis. \textcolor{red}{Add quantitative statement about magnification with LSST.} Importantly, for any of these surveys, the opportunity to perform magnification studies at all is a free source of additional lensing information. Regardless of whether the final deliberation for these surveys is to improve contamination modeling, to use only unaffected redshift regimes, or even to focus on alternative measures of magnification using sizes or redshift distributions, magnification information can and will be exploited, because it is free information. The work contained in this thesis has played an important role in laying the foundation for future studies that will maximize the use of weak gravitational lensing survey data.


%%%%%%%%%%%%%%%%%%%%%%%%%%%%%%%%%%%%%%%%%%%%%%%%%%%%%%%%%%%%%%%%%%%%%%
\endinput
Any text after an \endinput is ignored.
You could put scraps here or things in progress.
