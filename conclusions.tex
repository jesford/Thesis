%% The following is a directive for TeXShop to indicate the main file
%%!TEX root = diss.tex

\chapter{Conclusions}
\label{ch:conc}


%%%%%%%%%%%%%%%%%%%%%%%%%%%%%%%%%%%%%%%%%%%%%%%%%%%%%%%%%%%%%%%%%%%%%%
\section{Summary and Conclusions}
\label{sec:summary}

This thesis has presented novel results in the study of galaxy clusters through the use of weak gravitational lensing techniques. The scientific domain is the field of cosmology, which was discussed in \autoref{sec:Cosmology}. Within this field, the specific importance and usefulness of galaxy cluster studies was outlined in \autoref{sec:Clusters}. The main goal was to improve measurements of clusters masses, which are so important for both cosmological and astrophysical purposes. The methods used for improving galaxy cluster mass estimates consisted of two different and complementary approaches to weak gravitational lensing,. 

The first of these, weak lensing magnification, was explained in \autoref{sec:Mag} and is the dominant focus of the thesis. Magnification is a relatively underused technique, relative to the second component which is weak lensing shear, and deserved a thorough systematic investigation. Both types of weak lensing are statistical in nature, visible undetectable, and only measurable using many thousands of background galaxies behind many galaxy clusters or other gravitational lenses. Shear, which was explained in \autoref{sec:Shear}, is ubiquitous in the weak lensing literature, and involves measuring slight distortions in image shapes. Magnification can be measured in several different ways, but the one explored in this thesis uses the lensing-induced modifications to the source number densities that are detectable behind a gravitational lens.

%%%%%%%%%%%%%%%%%%%%%%%%%%%%%%%%%%%%%%%%%%%%%%%%%%%%%%%%%%%%%%%%%%%%%%
\section{Future Work}
\label{sec:future}


%%%%%%%%%%%%%%%%%%%%%%%%%%%%%%%%%%%%%%%%%%%%%%%%%%%%%%%%%%%%%%%%%%%%%%
\endinput
Any text after an \endinput is ignored.
You could put scraps here or things in progress.
