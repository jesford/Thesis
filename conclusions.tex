%% The following is a directive for TeXShop to indicate the main file
%%!TEX root = diss.tex

\chapter{Conclusions}
\label{ch:conc}


%%%%%%%%%%%%%%%%%%%%%%%%%%%%%%%%%%%%%%%%%%%%%%%%%%%%%%%%%%%%%%%%%%%%%%
\section{Thesis Summary}
\label{sec:summary}

This thesis has presented novel results in the study of galaxy clusters through the use of weak gravitational lensing techniques. The scientific domain is the field of cosmology, which was discussed in \autoref{sec:Cosmology}. Cosmologists seek to understand the nature of the universe on the largest scales -- the history, evolution, and fate, of the universe, its large scale structure, and all the components of its energy density. Part of this field, therefore requires knowledge of the matter distribution in the universe, and how that has changed over cosmic time. 

Matter makes up a significance portion of the universe ($\sim30$\% today), and almost all of it is composed of dark matter which appears to interact only through the gravitational force (see \autoref{sec:DM}). Matter was the dominant component of the universe for much of its history -- from the time of matter-radiation equality at $a(t) \approx 2.8 \times 10^{-4}$ until the transition to the current dark energy domination at $a(t) \approx 0.75$ \citep{RydenText}. A very useful way to probe the dark matter distribution is through the study of gravitationally collapsed halos, such as those that house galaxies and galaxy clusters.

The specific importance and usefulness of galaxy cluster studies was discussed in \autoref{sec:Clusters}. The main goal, regarding galaxy clusters in this thesis, was to improve measurements of clusters masses, which are so important for both cosmological and astrophysical purposes. The methods employed in this thesis for improving galaxy cluster mass estimates consisted of two different and complementary approaches to weak gravitational lensing. Weak lensing is the very subtle subfield of gravitational lensing (see \autoref{sec:Lensing}), wherein light from background sources is bent by gravitational potentials along its path, leading to the focusing and distortion of background galaxies. Unlike strong lensing, where the lensing-induced features like giant arcs and multiple images are usually visible to the eye in images, weak lensing is not strong enough to produce obvious effects.

The first of the two weak lensing techniques, weak lensing magnification, was explained in \autoref{sec:Mag} and is the dominant focus of the thesis. Magnification is a relatively underused technique, compared to the second component -- weak lensing shear, and deserved a thorough systematic investigation. Both types of weak lensing are statistical in nature, only measurable using many thousands of background galaxies behind many galaxy clusters or other gravitational lenses. Shear, which was explained in \autoref{sec:Shear}, is ubiquitous in the weak lensing literature, and involves measuring slight distortions in image shapes. Magnification can be measured in several different ways, but the one explored in this thesis uses the lensing-induced modifications to the source number densities that are detectable behind a gravitational lens.

The body of this thesis was composed of three published articles, modified into the form required herein, and sandwiched between an introduction to the relevant topics in \autoref{ch:Introduction}, and the current concluding chapter. These three peer-reviewed journal articles were all scientifically-led by the thesis author. The publication details can be found either in the Thesis Preface or in the Bibliography references for \citet{Ford12}, \citet{Ford14}, and \citet{Ford15}. 

\autoref{ch2} was comprised of the first magnification study applied to galaxy groups. Prior magnification studies were of galaxy lenses or of very massive (i.e. strong lensing) clusters. This study yielded the first test of the technique on intermediate mass scales (several $10^{13} M_{\odot}$), and also included the first direct comparison between shear-measured masses and magnification-measured masses. These mass estimates were found to be in agreement, albeit with rather large uncertainties. The signal-to-noise from each technique was quantified and compared -- magnification yielded a 4.8$\sigma$ detection, whereas shear achieved 11$\sigma$.

In \autoref{ch3} and \autoref{ch4}, a much larger galaxy clusters sample was explored -- the \ac{3D-MF} cluster catalogue from the \ac{CFHTLS}-Wide fields. This is one of the largest galaxy cluster catalogues that has ever been compiled, notable containing many lower mass galaxy groups, and covering a wide range of redshifts, up to $z \sim 1$. Optimized with the goal of finding as many clusters as possible, the \ac{3D-MF} cluster finder produced a catalogue that is 100\% complete for clusters with mass greater than $3 \times 10^{14} M_{\odot}$ and 88\% complete above $10^{14} M_{\odot}$ \citep[see \autoref{3DMF} or the original \ac{3D-MF} paper][for more details]{Milkeraitis10}. Compared with the small sample of X-ray selected galaxy groups analyzed in \autoref{ch2}, the \ac{3D-MF} clusters in \autoref{ch3} and \autoref{ch4} allow for superior signal-to-noise even when the clusters are binned according to different attributes. This allows for very interesting studies of clusters as a function of redshift and cluster richness (number of galaxies).

\autoref{ch3} focuses on the magnification signal of the \ac{3D-MF} clusters, which was detected at a significance of $9.7\sigma$. This chapter described the first magnification analysis with a large enough cluster sample to be able to bin as a function of richness \citep[however, note that the publication of this work was followed in quick succession by a very interesting magnification study of clusters and luminous red galaxies in the \ac{SDSS} by][]{Bauer14}. Using the richness-binned cluster mass estimates, a power-law scaling between cluster mass and richness was determined. This work compared the goodness of fit for two different types of cluster miscentering models -- the model assuming that \ac{3D-MF}'s centers were perfectly accurate did not fit as well as the model that assumed an offset distribution based on simulations.

Additionally, the masses of clusters binned as a function of redshift were obtained, yielding unexpected variation of mass across redshift bins that had very similar richness distributions. This finding inspired further efforts to model a potential systematic effect for magnification -- physical overlap between lenses and sources in redshift space. Assuming some fraction of source contamination, the expected physical clustering signal was calculated and included in the models that were fit to the magnification signal. This was the first time such an approach had been attempted -- previous magnification studies have chosen to simply avoid interpreting any measurements in redshift regions where contamination is expected to be significant.

\autoref{ch4} paralleled the previous chapter in many respects. This included a shear analysis in the same spirit as the magnification one in \autoref{ch3}. The binning of clusters was identical in each analysis, in order to produce a side-by-side comparison of the results from each of the two techniques. At 54$\sigma$, the shear signal measured from the cluster lenses was very strong, and this study was able to go beyond what was possible with magnification. In addition to measuring the mass-richness scaling relation, this analysis also searched for any evidence of the relation's evolution with cluster redshift. No significant evolution was detected, which is in line with other recent studies \citep{Andreon14}. 

This work also placed constraints on the \ac{3D-MF} cluster centroid offsets, which was an improvement over the magnification analysis of \autoref{ch3} because shear is more sensitive to offset halos. Instead of comparing a perfectly centred and an offset model, the shear analysis actually fit for the distribution of offsets, finding that it to be reasonably well described by a Gaussian peaking at a radial offset of $\sigma_{\rm off} \sim 0.4$ arcmin. (Although see \autoref{richbintable1} and \autoref{ztable1} for exact offset values as well as other parameters). 

A thorough comparison of the results obtained for the \ac{3D-MF} clusters, with other cluster analyses in the literature was presented in \autoref{ch4:disc}. Briefly, the slope of the mass-richness relation determined in this thesis was in line with other similar work \citep{Wen12,Covone12}, while the normalization less meaningful because it is highly sensitive to the definition of richness used for the sample (which varies widely in the literature). The cluster miscentering offsets measured in \autoref{ch4} of this thesis were intermediate between other results in the literature, which ranged from about half the radial offsets of \ac{3D-MF} \citep{George12} to several times larger than \ac{3D-MF} \citep{Johnston07}.


%%%%%%%%%%%%%%%%%%%%%%%%%%%%%%%%%%%%%%%%%%%%%%%%%%%%%%%%%%%%%%%%%%%%%%
\section{Final Conclusions}
\label{sec:conc}

The overall important findings produced by this thesis research can be summarized as a list of key take-away messages:
\begin{itemize}
\item first thing
\item second thing
\end{itemize}

%%%%%%%%%%%%%%%%%%%%%%%%%%%%%%%%%%%%%%%%%%%%%%%%%%%%%%%%%%%%%%%%%%%%%%
\section{Future Prospects}
\label{sec:future}

\textcolor{red}{compare 3dmf clusters to other cluster catalogues in or overlapping the WIDE; run cluster finders on same optical dataset; miscentering follow up using different centres; xray follow up; cluster classification according to morphology; public code to make miscentering and other aspects of this work easier to reproduce and improve upon; future of magnification... need to know z very well or better model contamination; other magnification systematics Hendrik is investigating?; hybrid lensing incorporating all info; future surveys like LSST, Euclid, others? and details about how magnification might benefit them}

%%%%%%%%%%%%%%%%%%%%%%%%%%%%%%%%%%%%%%%%%%%%%%%%%%%%%%%%%%%%%%%%%%%%%%
\endinput
Any text after an \endinput is ignored.
You could put scraps here or things in progress.
