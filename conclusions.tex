%% The following is a directive for TeXShop to indicate the main file
%%!TEX root = diss.tex

\chapter{Conclusions}
\label{ch:conc}


%%%%%%%%%%%%%%%%%%%%%%%%%%%%%%%%%%%%%%%%%%%%%%%%%%%%%%%%%%%%%%%%%%%%%%
\section{Summary}
\label{sec:summary}

This thesis has presented novel results in the study of galaxy clusters through the use of weak gravitational lensing techniques. The scientific domain is the field of cosmology, which was discussed in \autoref{sec:Cosmology}. Cosmologists seek to understand the nature of the universe on the largest scales -- the history, evolution, and fate, of the universe, its large scale structure, and all the components of its energy density. Part of this field, therefore requires knowledge of the matter distribution in the universe, and how that has changed over cosmic time. 

Matter makes up a significance portion of the universe ($\sim 30$\% today), and almost all of it is composed of dark matter which appears to interact only through the gravitational force (see \autoref{sec:DM}). Matter was the dominant component of the universe for much of its history -- from the time of matter-radiation equality at $a(t) \approx 2.8 \times 10^{-4}$ until the transition to the current dark energy domination at $a(t) \approx 0.75$ \citep{RydenText}. A very useful way to probe the dark matter distribution is through the study of gravitationally collapsed halos, such as those that house galaxies and galaxy clusters.

The specific importance and usefulness of galaxy cluster studies was discussed in \autoref{sec:Clusters}. The main goal, regarding galaxy clusters in this thesis, was to improve measurements of clusters masses, which are so important for both cosmological and astrophysical purposes. The methods employed in this thesis for improving galaxy cluster mass estimates consisted of two different and complementary approaches to weak gravitational lensing. Weak lensing is the very subtle subfield of gravitational lensing (see \autoref{sec:Lensing}), wherein light from background sources is bent by gravitational potentials along its path, leading to the focusing and distortion of background galaxies. Unlike strong lensing, where the lensing-induced features like giant arcs and multiple images are usually visible to the eye in images, weak lensing is not strong enough to produce obvious effects.

The first of the two weak lensing techniques, weak lensing magnification, was explained in \autoref{sec:Mag} and is the dominant focus of the thesis. Magnification is a relatively underused technique, compared to the second component -- weak lensing shear, and deserved a thorough systematic investigation. Both types of weak lensing are statistical in nature, only measurable using many thousands of background galaxies behind many galaxy clusters or other gravitational lenses. Shear, which was explained in \autoref{sec:Shear}, is ubiquitous in the weak lensing literature, and involves measuring slight distortions in image shapes. Magnification can be measured in several different ways, but the one explored in this thesis uses the lensing-induced modifications to the source number densities that are detectable behind a gravitational lens.

The body of this thesis was composed of three published articles, modified into the form required herein, and sandwiched between an introduction to the relevant topics in \autoref{ch:Introduction}, and the current concluding chapter. These three peer-reviewed journal articles were all scientifically-led by the thesis author. The publication details can be found either in the Thesis Preface or in the Bibliography references for \citet{Ford12}, \citet{Ford14}, and \citet{Ford15}. 

\autoref{ch2} was comprised of the first magnification study applied to galaxy groups. Prior magnification studies were of galaxy lenses or of very massive (i.e. strong lensing) clusters. This study yielded the first test of the technique on intermediate mass scales (several $10^{13} \M_{odot}$), and also included the first direct comparison between shear-measured masses and magnification-measured masses. These mass estimates were found to be in agreement, albeit with rather large uncertainties. The signal-to-noise from each technique was quantified and compared -- magnification yielded a 4.8$\sigma$ detection, whereas shear achieved 11$\sigma$.

In \autoref{ch3} and \autoref{ch4}, a much larger galaxy clusters sample was explored -- the \ac{3D-MF} cluster catalogue from CFHTLenS


%%%%%%%%%%%%%%%%%%%%%%%%%%%%%%%%%%%%%%%%%%%%%%%%%%%%%%%%%%%%%%%%%%%%%%
\section{Final Conclusions}
\label{sec:conc}

The overall important findings produced by this thesis research can be summarized as a list of key take-away messages:
\begin{itemize}
\item first thing
\item second thing
\end{itemize}

%%%%%%%%%%%%%%%%%%%%%%%%%%%%%%%%%%%%%%%%%%%%%%%%%%%%%%%%%%%%%%%%%%%%%%
\section{Future Prospects}
\label{sec:future}


%%%%%%%%%%%%%%%%%%%%%%%%%%%%%%%%%%%%%%%%%%%%%%%%%%%%%%%%%%%%%%%%%%%%%%
\endinput
Any text after an \endinput is ignored.
You could put scraps here or things in progress.
