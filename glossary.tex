%% The following is a directive for TeXShop to indicate the main file
%%!TEX root = diss.tex

\chapter{Glossary}

%This glossary uses the handy \latexpackage{acroynym} package to automatically maintain the glossary.  It uses the package's \texttt{printonlyused} option to include only those acronyms explicitly referenced in the \LaTeX\ source.

% use \acrodef to define an acronym, but no listing
\acrodef{UI}{user interface}
\acrodef{UBC}{University of British Columbia}

% The acronym environment will typeset only those acronyms that were
% *actually used* in the course of the document

\begin{acronym}[CFHTLenS] %longest acronym used, for formatting list

%JES' ACRONYMS
\acro{CFHT}{Canada-France-Hawaii Telescope\acroextra{. Description.}}
\acro{CFHTLenS}[CFHTL${\rm en}$S]{Canada-France-Hawaii Telescope Lensing Survey\acroextra{. Description.}}%{\LU{C}{c}anada \LU{C}{c}anada \LU{C}{c}anada}
%\acro{CFHTLenS}{Canada-France-Hawaii-Telescope Lensing Survey}
\acro{CFHTLS}{Canada-France-Hawaii Telescope Legacy Survey\acroextra{. Description.}}
\acro{CMB}{Cosmic Microwave Background\acroextra{. Description.}}
\acro{COBE}{COsmic Background Explorer\acroextra{. Description.}}
\acro{COSMOS}{Cosmological Evolution Survey\acroextra{. Description.}}
%\acro{Euclid}{Euclid\acroextra{. Description.}}
\acro{LBG}{Lyman-break galaxy\acroextra{. Description.}}
\acro{LF}{Luminosity Function\acroextra{. Description.}}
%\acro{LSST}{Large Synoptic Survey Telescope\acroextra{. Description.}}
\acro{MACHO}{MAssive Compact Halo Object\acroextra{. Description.}}
\acro{NFW}{Navarro-Frenk-White\acroextra{. Description.}}
\acro{SDSS}{Sloan Digital Sky Survey\acroextra{. Description.}}
\acro{SIS}{Singular Isothermal Sphere\acroextra{. Description.}}
%\acro{WFIRST}{Wide-Field InfraRed Survey Telescope\acroextra{. Description.}}
\acro{WIMP}{Weakly Interacting Massive Particle\acroextra{. Description.}}
\acro{WMAP}{Wilkinson Microwave Anisotropy Probe\acroextra{. Description.}}
\acro{3D-MF}{3D-Matched-Filter\acroextra{. Description.}}

\end{acronym}

%\vspace{1cm}

\section*{\underline{Symbols}}

\begin{tabular}{ll}
$a(t)$ & Scale factor of the universe, defined to be unity today. \\
$G$ & Newton's gravitational constant. $6.673 \times 10^{-11}$ N m$^2$ kg$^{-2}$.\\
$c$ & Speed of light in a vacuum. $2.998 \times 10^8$ m s$^{-1}$. \\
$z$ & Redshift, a common distance measure in cosmology. \\
\end{tabular}

\section*{\underline{Units}}

\begin{tabular}{ll}
Mpc & Megaparsec. A unit of distance in cosmology, equal to 10$^6$ parsecs, where a parsec is $3.086 \times 10^{16}$ meters. \\
second & description \\
and & so on \\
\end{tabular}




% You can also use \newacro{}{} to only define acronyms
% but without explictly creating a glossary
% 
% \newacro{ANOVA}[ANOVA]{Analysis of Variance\acroextra{, a set of
%   statistical techniques to identify sources of variability between groups.}}
% \newacro{API}[API]{application programming interface}
% \newacro{GOMS}[GOMS]{Goals, Operators, Methods, and Selection\acroextra{,
%   a framework for usability analysis.}}
% \newacro{TLX}[TLX]{Task Load Index\acroextra{, an instrument for gauging
%   the subjective mental workload experienced by a human in performing
%   a task.}}
% \newacro{UI}[UI]{user interface}
% \newacro{UML}[UML]{Unified Modelling Language}
% \newacro{W3C}[W3C]{World Wide Web Consortium}
%\acro{XML}[XML]{Extensible Markup Language}
