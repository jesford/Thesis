%% The following is a directive for TeXShop to indicate the main file
%%!TEX root = diss.tex

\chapter{Glossary}

%This glossary uses the handy \latexpackage{acroynym} package to automatically maintain the glossary.  It uses the package's \texttt{printonlyused} option to include only those acronyms explicitly referenced in the \LaTeX\ source.

% use \acrodef to define an acronym, but no listing
\acrodef{UI}{user interface}
\acrodef{UBC}{University of British Columbia}

% The acronym environment will typeset only those acronyms that were
% *actually used* in the course of the document

\begin{acronym}[CFHTLenS] %longest acronym used, for formatting list

%JES' ACRONYMS
\acro{CFHT}{Canada-France-Hawaii Telescope\acroextra{. A 3.6m optical/infrared telescope on the summit of Mauna Kea, in Hawaii.}}
\acro{CFHTLenS}[CFHTL${\rm en}$S]{Canada-France-Hawaii Telescope Lensing Survey\acroextra{. The gravitational lensing survey incorporating the \acs{CFHTLS} data. The \acs{CFHTLenS} team has produced the only publicly available shear catalog to date.}}%{\LU{C}{c}anada \LU{C}{c}anada \LU{C}{c}anada}
%\acro{CFHTLenS}{Canada-France-Hawaii-Telescope Lensing Survey}
\acro{CFHTLS}{Canada-France-Hawaii Telescope Legacy Survey\acroextra{. Astronomical survey optimized for weak lensing, covering 154 deg$^2$ of sky in 5 filters covering the optical to near-infrared wavelengths.}}
\acro{CMB}{Cosmic Microwave Background\acroextra{. Thermal radiation consisting of photons that were released from Thomson scattering when neutral atoms first formed in the early universe.}}
\acro{COBE}{COsmic Background Explorer\acroextra{. The first satellite dedicated to \acs{CMB} measurements, launched in 1989.}}
\acro{COSMOS}{Cosmological Evolution Survey\acroextra{. A very deep 2 deg$^2$ survey aimed at cosmological studies, incorporating data from many different space and ground-based telescopes (including Hubble, Spitzer, Chandra, and many others).}}
%\acro{Euclid}{Euclid\acroextra{. Description.}}
\acro{LBG}{Lyman-break galaxy\acroextra{. High-redshift star forming galaxy that emits only at wavelengths longer than the Lyman limit (rest frame 912\AA), allowing detection through observations in multiple filters.}}
\acro{LF}{Luminosity Function\acroextra{. A functional description of the number of objects (e.g. galaxies) as a function of luminosity or magnitude.}}
%\acro{LSST}{Large Synoptic Survey Telescope\acroextra{. Description.}}
\acro{MACHO}{MAssive Compact Halo Object\acroextra{. A term encompassing black holes, planets, and other compact objects, \acs{MACHO}s were once a serious dark matter candidate.}}
\acro{NFW}{Navarro-Frenk-White\acroextra{. A halo mass model based on simulations of dissipationless gravitational collapse. Density is proportional to $1/r$ for small radii, transitioning to $1/r^3$ for large radii.}}
\acro{Planck}[P${\rm lanck}$]{Planck\acroextra{. The most recent generation of \acs{CMB} satellites, providing the tightest cosmological constraints to date.}}
\acro{SDSS}{Sloan Digital Sky Survey\acroextra{. Ongoing multicolor imaging and spectroscopic survey of one third of the sky, using a 2.5m telescope at Apache Point, New Mexico.}}
\acro{SIS}{Singular Isothermal Sphere\acroextra{. A very simple model of halo mass, with density proportional to $1/r^2$.}}
%\acro{WFIRST}{Wide-Field InfraRed Survey Telescope\acroextra{. Description.}}
\acro{WIMP}{Weakly Interacting Massive Particle\acroextra{. A popular theoretical candidate for the dark matter particle.}}
\acro{WMAP}{Wilkinson Microwave Anisotropy Probe\acroextra{. The second generation of \acs{CMB} satellites, \acs{WMAP} created detailed maps of temperature fluctuations in the early universe.}}
\acro{3D-MF}{3D-Matched-Filter\acroextra{. Optical galaxy cluster finding algorithm, which recovered over 18,000 cluster candidates in the \acs{CFHTLS}.}}

\end{acronym}

%\vspace{1cm}

\section*{\underline{Symbols}}

\begin{tabular}{ll}
$a(t)$ & Scale factor of the universe, defined to be unity today. \\
$G$ & Newton's gravitational constant. $G = 6.673 \times 10^{-11}$ N m$^2$ kg$^{-2}$.\\
$c$ & Speed of light in a vacuum. $c = 2.998 \times 10^8$ m s$^{-1}$. \\
$H(z)$ & Hubble rate. The redshift (or time) dependent expansion rate of the universe. \\
$H_0$ & Hubble constant. The present-day Hubble rate. $H_0 \sim 67$ km s$^{-1}$ Mpc$^{-1}$. \\
$\Omega_{\rm m}$ & Matter fraction. The fraction of the universe consisting of matter. $\Omega_{\rm m} \sim 0.3$ \\
$\Omega_{\rm \Lambda}$ & Dark energy fraction. The fraction of the universe consisting of dark energy. $\Omega_{\rm \Lambda} \sim 0.7$. \\
$\sigma_8$ & Normalization of the matter power spectrum.  $\sigma_8 \sim 0.8$.\\
$z$ & Redshift. A common distance measure in cosmology. \\
$\rho_{\rm crit}(z)$ & Critical energy density of the universe. Current value is $\rho_{\rm crit}(0) = 1.88h^2 \times 10^{-29}$ g cm$^{-3}$. \\
$\mu$ & Magnification. Simply proportional to $\kappa$ in the weak lensing limit, $\mu$ is a function of the gravitational lens mass. \\
$\kappa$ & Convergence. The component of gravitational lensing describing the isotropic focusing of light rays. \\
$\gamma$ & Shear. The two component pseudo-vector quantifying the anisotropic focusing of gravitationally lensed light rays. \\
$\alpha(m)$ & \acs{LF} slope. The logarithmic slope of the source luminosity function, controls the direction of the number count effect of magnification. \\
\end{tabular}

\section*{\underline{Units}}

\begin{tabular}{ll}
Mpc & Megaparsec. A unit of distance in cosmology, equal to 10$^6$ parsecs, where a parsec is $3.086 \times 10^{16}$ m. \\
$M_{\odot}$ & Solar mass. A common measure of mass in astronomy, equal to $1.989 \times 10^{30}$ kg. \\
and & so on. \\
\end{tabular}




% You can also use \newacro{}{} to only define acronyms
% but without explictly creating a glossary
% 
% \newacro{ANOVA}[ANOVA]{Analysis of Variance\acroextra{, a set of
%   statistical techniques to identify sources of variability between groups.}}
% \newacro{API}[API]{application programming interface}
% \newacro{GOMS}[GOMS]{Goals, Operators, Methods, and Selection\acroextra{,
%   a framework for usability analysis.}}
% \newacro{TLX}[TLX]{Task Load Index\acroextra{, an instrument for gauging
%   the subjective mental workload experienced by a human in performing
%   a task.}}
% \newacro{UI}[UI]{user interface}
% \newacro{UML}[UML]{Unified Modelling Language}
% \newacro{W3C}[W3C]{World Wide Web Consortium}
%\acro{XML}[XML]{Extensible Markup Language}
