%% The following is a directive for TeXShop to indicate the main file
%%!TEX root = diss.tex

\chapter{Glossary}

%This glossary uses the handy \latexpackage{acroynym} package to automatically maintain the glossary.  It uses the package's \texttt{printonlyused} option to include only those acronyms explicitly referenced in the \LaTeX\ source.

% use \acrodef to define an acronym, but no listing
\acrodef{UI}{user interface}
\acrodef{UBC}{University of British Columbia}

% The acronym environment will typeset only those acronyms that were
% *actually used* in the course of the document

\begin{acronym}[CFHTLenS] %longest acronym used, for formatting list

%JES' ACRONYMS
\acro{AGN}{Active Galactic Nuclei\acroextra{.}}
\acro{BCG}{Brightest Cluster Galaxy\acroextra{. In many rich clusters, this obvious and very bright elliptical galaxy sits at the center of the gravitational potential (the acronym is sometimes interpreted as Brightest Central Galaxy).}}
\acro{CFHT}{Canada-France-Hawaii Telescope\acroextra{. A 3.6m optical/infrared telescope on the summit of Mauna Kea, in Hawaii.}}
\acro{CFHTLenS}[CFHTL${\rm en}$S]{Canada-France-Hawaii Telescope Lensing Survey\acroextra{. The gravitational lensing survey incorporating the \acs{CFHTLS} data. The \acs{CFHTLenS} team has produced the only publicly available shear catalog to date.}}%{\LU{C}{c}anada \LU{C}{c}anada \LU{C}{c}anada}
%\acro{CFHTLenS}{Canada-France-Hawaii-Telescope Lensing Survey}
\acro{CFHTLS}{Canada-France-Hawaii Telescope Legacy Survey\acroextra{. Astronomical survey optimized for weak lensing, covering 154 deg$^2$ of sky in 5 filters covering the optical to near-infrared wavelengths.}}
\acro{CMB}{Cosmic Microwave Background\acroextra{. Thermal radiation consisting of photons that were released from Thomson scattering when neutral atoms first formed in the early universe.}}
\acro{COBE}{COsmic Background Explorer\acroextra{. The first satellite dedicated to \acs{CMB} measurements, launched in 1989.}}
\acro{COSMOS}{Cosmological Evolution Survey\acroextra{. A very deep 2 deg$^2$ survey aimed at cosmological studies, incorporating data from many different space and ground-based telescopes (including Hubble, Spitzer, Chandra, and many others).}}
\acro{Euclid}[E${\rm uclid}$]{Euclid\acroextra{. A planned 1.2 meter space-based telescope which will rely partly on weak lensing to achieve its main goal of constraining dark energy.}}
\acro{ICM}{Intracluster Medium\acroextra{. The gas and ionized plasma that fills the space between galaxies in a cluster.}}
\acro{LBG}{Lyman-break galaxy\acroextra{. High-redshift star forming galaxy that emits only at wavelengths longer than the Lyman limit (rest frame 912\AA), allowing detection through observations in multiple filters.}}
\acro{LF}{Luminosity Function\acroextra{. The number of objects (e.g. galaxies) as a function of luminosity or magnitude.}}
\acro{LSST}{Large Synoptic Survey Telescope\acroextra{. An 8.4 meter ground-based telescope, currently under construction in Chile, with wide ranging science goals including weak lensing.}}
\acro{MACHO}{MAssive Compact Halo Object\acroextra{. A term encompassing black holes, planets, and other compact objects, \acs{MACHO}s were once a serious dark matter candidate.}}
\acro{NFW}{Navarro-Frenk-White\acroextra{. A halo mass model based on simulations of dissipationless gravitational collapse. Density is proportional to $1/r$ for small radii, transitioning to $1/r^3$ for large radii.}}
\acro{Planck}[P${\rm lanck}$]{Planck\acroextra{. The most recent generation of \acs{CMB} satellites, providing the tightest cosmological constraints to date.}}
\acro{SDSS}{Sloan Digital Sky Survey\acroextra{. Ongoing multicolor imaging and spectroscopic survey of one third of the sky, using a 2.5m telescope at Apache Point, New Mexico.}}
\acro{SIS}{Singular Isothermal Sphere\acroextra{. A very simple model of halo mass, with density proportional to $1/r^2$.}}
\acro{WFIRST}{Wide-Field InfraRed Survey Telescope\acroextra{. A planned 2.4 meter space-based telescope, with wide ranging science goals including weak lensing.}}
\acro{WIMP}{Weakly Interacting Massive Particle\acroextra{. A popular theoretical candidate for the dark matter particle.}}
\acro{WMAP}{Wilkinson Microwave Anisotropy Probe\acroextra{. The second generation of \acs{CMB} satellites, \acs{WMAP} created detailed maps of temperature fluctuations in the early universe.}}
\acro{3D-MF}{3D-Matched-Filter\acroextra{. Optical galaxy cluster finding algorithm, which recovered over 18,000 cluster candidates in the \acs{CFHTLS}.}}

\end{acronym}

%\vspace{0.2cm}

\section*{\underline{Units}}
\begin{tabular}{p{0.6in}p{5.8in}}
arcmin & Arcminute. Unit of angular separation on the sky. 1 degree = 60 arcmin. \\
$M_{\odot}$ & Solar mass. A common measure of mass in astronomy, equal to $1.989 \times 10^{30}$ kg. \\
pc & Parsec. A unit of distance in astronomy, equal to $3.086 \times 10^{16}$ m. Mpc = 10$^6$ pc, kpc = 10$^3$ pc. \\
\end{tabular}

\vspace{0.2cm}

\section*{\underline{Symbols}}
\begin{tabular}{p{0.6in}p{5.8in}}

$a(t)$ & Scale factor of the universe, defined to be unity today. \\
$\cal{A}$ & Amplification matrix, describing the gravitational lens mapping from source to image plane. \\
$\alpha(m)$ & \acs{LF} slope. The logarithmic slope of the source luminosity function, controls the direction of the number count effect of magnification. \\
$b$ & Bias factor. Dimensionless number quantifying how much an object clusters relative to the dark matter in the universe. \\ 
$\beta$ & Slope of the mass-richness power-law relation. \\
$c$ & Speed of light in a vacuum. $c = 2.998 \times 10^8$ m s$^{-1}$. \\
$d_A$ & Angular diameter distance. The cosmological distance equal to length divided by angle subtended. \\

\end{tabular}
\section*{\underline{}}
\vspace{-0.5cm}
\begin{tabular}{p{0.6in}p{5.8in}}

$d_L$ & Luminosity distance. The cosmological distance for which flux drops as luminosity over (distance)$^2$. \\
$d_p$ & Proper distance. The cosmological distance that would be measured between two objects if you could lay down a very large ruler. \\
$D_{\rm l}$ & Angular diameter distance from the observer to the gravitational lens. \\
$D_{\rm ls}$ & Angular diameter distance between the gravitational lens and source. \\
$D_{\rm s}$ & Angular diameter distance from the observer to the gravitationally lensed source. \\
$D(z)$ & Linear Growth Function. \\
$\delta\mu$ & Magnification contrast. $\delta\mu \equiv \mu -1$. \\
$\delta_{\rm c}$ & Characteristic overdensity of a halo. An \acs{NFW} parameter that is a function of $c_{200}$. \\
$f_{\rm clustering}$ & Fraction of the background source sample that is at the same redshift as the lenses, and therefore leads to a clustering signal. \\
$f_{\rm lensing}$ & Fraction of the background source sample that is at high redshift, and therefore can be lensed. \\
$G$ & Newton's gravitational constant. $G = 6.673 \times 10^{-11}$ N m$^2$ kg$^{-2}$.\\
$g$ & Reduced shear. $g = \gamma/(1-\kappa)$. \\
$\gamma$ & Shear. The two component pseudo-vector quantifying the anisotropic focusing of gravitationally lensed light rays. \\
$\gamma_{\rm t}$ & Tangential shear. Component of the shear oriented tangential to the direction of the lens. \\
$h$ & Hubble parameter. $h \equiv H_0/(100\ $km s$^{-1}$ Mpc$^{-1}$). \\
$H(z)$ & Hubble rate. The redshift (or time) dependent expansion rate of the universe. \\
$H_0$ & Hubble constant. The present-day Hubble rate. $H_0 \sim 67$ km s$^{-1}$ Mpc$^{-1}$. \\
$\kappa$ & Convergence. The part of gravitational lensing composed of the isotropic focusing of light rays. \\
$m$ & Apparent magnitude. Logarithmic measure of flux relative to a reference value. \\
$M$ & Absolute magnitude. Equal to the apparent magnitude a source would have if it was located at a luminosity distance of 10 pc. \\
$M_0$ & Normalization of the mass-richness relation, defined to be the average mass of clusters with richness $N_{200}$. \\
$M_{200}$ & Mass of a dark matter halo within the radius $R_{200}$. \\
$\mu$ & Magnification. Simply proportional to $\kappa$ in the weak lensing limit, $\mu$ is a function of the gravitational lens mass. \\
$n$ & Observed source number counts as a function of magnitude or flux and redshift. \\
$n_0$ & Unlensed (intrinsic) source number counts. \\
$N_{200}$ & Richness. The number of galaxies brighter than $i$-band absolute magnitude -19.35, within the radius $R_{200}$. \\
$\omega$ & Size of a galaxy image. \\
$\Omega_{\rm b}$ & Baryon density parameter. The fraction of the universe consisting of baryons. $\Omega_{\rm b} \sim 0.05$. \\
$\Omega_{\rm c}$ & Cold dark matter density parameter. The fraction of the universe consisting of dark matter. $\Omega_{\rm c} \sim 0.27$. \\
$\Omega_{\Lambda}$ & Dark energy density parameter. The fraction of the universe consisting of dark energy. $\Omega_{\rm \Lambda} \sim 0.7$. \\
$\Omega_{\rm m}$ & Matter density parameter. The fraction of the universe consisting of matter, including dark matter and baryons. $\Omega_{\rm m} \sim 0.3$. \\

\end{tabular}
\section*{\underline{}}
\vspace{-0.5cm}
\begin{tabular}{p{0.6in}p{5.8in}}

$\Omega_{\rm r}$ & Radiation density parameter. The fraction of the universe consisting of relativistic particles. $\Omega_{\rm r} \sim 8 \times 10^{-5}$. \\

$P(k)$ & Power spectrum. \\
$P(R_{\rm off})$ & Probability distribution of cluster centroid offsets. \\
$P(z)$ & Probability distribution of redshifts. \\
$p_{\rm cc}$ & Fraction of \acs{3D-MF} clusters that are correctly centered on their dark matter halos. \\
$\Phi(M)$ & Schechter function. A common parameterization of the \acs{LF}. \\
$r$ & Comoving distance, that grows along with the expansion of space. \\
$R$ & Projected physical distance on the sky. \\
$R_{200}$ & Radius of a dark matter halo (at $z$) within which the average density is 200$\rho_{\rm crit}(z)$. \\
$R_{\rm off}$ & The distance between the \acs{3D-MF} identified center of a cluster, and the true center of a dark matter halo. \\
$\rho_{\rm crit}(z)$ & Critical energy density of the universe. Current value is $\rho_{\rm crit}(0) \sim 9.2 \times 10^{-27}$ kg m$^{-3}$. \\
$\Sigma$ & Surface mass density. Mass density projected onto the 2D plane of the sky. \\
$\Sigma_{\rm crit}$ & Critical surface mass density. A function of the angular diameter distances between observer, lens and source, this is the minimum surface mass density of a lens for it to produce strong lensing features. \\
$\Sigma^{\rm sm}$ & Smoothed surface mass density of a stack of clusters that are offset as described by the distribution $P(R_{\rm off})$. \\
$\Sigma_{\rm 2halo}$ & The two halo term. The contribution to the surface mass density coming from nearby structure on the sky. \\
$\Delta\Sigma$ & Differential surface mass density. Defined as the difference between the surface mass density at some radius from the center of a lens, and the average inside of that radius. \\
$\sigma_{\rm cl}$ & The detection significance of a \acs{3D-MF} cluster candidate. \\
$\sigma_{\rm off}$ & The radius where the $P(R_{\rm off})$ distribution peaks. \\
$\sigma_8$ & Normalization of the matter power spectrum.  $\sigma_8 \sim 0.8$. \\
$\psi(\theta)$ & Two-dimensional analogue to the Newtonian gravitational potential. \\
$t_{\rm e}$ & Time of emission of light. \\
$t_0$ & Present time, current age of the universe. \\
$w$ & Correlation function. \\
$w_{\rm dm}$ & Auto-correlation function of dark matter. \\
$w_{\rm opt}$ & Optimally-weighted correlation function for magnification. \\
$\chi^2$ & Chi-squared statistic, quantifying the goodness-of-fit of a model to the data. \\
$\chi^2_{\rm red}$ & Reduced chi-squared statistic. $\chi^2_{\rm red}$ is $\chi^2$ divided by the number of degrees of freedom in the model. \\
$z$ & Redshift. A common distance measure in cosmology. \\

\end{tabular}



% You can also use \newacro{}{} to only define acronyms
% but without explictly creating a glossary
% 
% \newacro{ANOVA}[ANOVA]{Analysis of Variance\acroextra{, a set of
%   statistical techniques to identify sources of variability between groups.}}
% \newacro{API}[API]{application programming interface}
% \newacro{GOMS}[GOMS]{Goals, Operators, Methods, and Selection\acroextra{,
%   a framework for usability analysis.}}
% \newacro{TLX}[TLX]{Task Load Index\acroextra{, an instrument for gauging
%   the subjective mental workload experienced by a human in performing
%   a task.}}
% \newacro{UI}[UI]{user interface}
% \newacro{UML}[UML]{Unified Modelling Language}
% \newacro{W3C}[W3C]{World Wide Web Consortium}
%\acro{XML}[XML]{Extensible Markup Language}
