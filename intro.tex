%% The following is a directive for TeXShop to indicate the main file
%%!TEX root = diss.tex

\chapter{Introduction}
\label{ch:Introduction}

%\begin{epigraph}
%   \emph{If I have seen farther it is by standing on the shoulders of Giants.} ---~Sir Isaac Newton (1855)
%\end{epigraph}

%This first paragraph will give a little intro and describe what is coming up in \autoref{sec:Cosmology}, \autoref{sec:Lensing} and \autoref{sec:Clusters}. The \acf{CFHTLenS} was a big lensing project in the \acf{CFHTLS} survey. This is an example of an acronym that does not go in the table of acronyms:  \acs{UBC}. 

This thesis is concerned with weak gravitational lensing studies of galaxy clusters, using two complementary approaches, shear and magnification. This introductory chapter provides the necessary background and context for understanding the novel research presented in the subsequent chapters. The basics of cosmology, which is the larger field within which this thesis research resides, is given in \autoref{sec:Cosmology}, with a particular focus on distances, which will be important for the presentation of gravitational lensing in \autoref{sec:Lensing}. Galaxy clusters are discussed in \autoref{sec:Clusters}, from the cosmological as well as the intracluster physics perspective. \autoref{sec:Impact} outlines the novelty and importance of the research contained in this thesis. A brief overview of the body of work which is presented in the main chapters of this thesis is given in \autoref{sec:Overview}.

%%%%%%%%%%%%%%%%%%%%%%%%%%%%%%%%%%%%%%%%%%%%%%%%%%%%%%%%%%%%%%%%%%%%%%
\section{Cosmology}
\label{sec:Cosmology}

Cosmology is the study of our universe as a whole. It can be easy to take for granted the simple fact that we, as scientists, can even do cosmology at all -- that is, we can make quantitative and testable predictions about the physical nature of the vast universe we inhabit. At the same time, as we cosmologists forge ahead, caught up in the day-to-day struggles that concern some minute detail of a model, a prediction, or an idea, it is easy to overlook the sheer beauty of what we are so deeply invested in. The introduction of this thesis serves both to lay the requisite theoretical foundation, upon which the thesis research relies, while also giving an honest depiction of the big picture ideas for which this work strives to be relevant.

"The Cosmos is all that is or was or ever will be. Our feeblest contemplations of the Cosmos stir us -- there is a tingling in the spine, a catch in the voice, a faint sensation, as if a distant memory, of falling from a height. We know we are approaching the greatest of mysteries" \citep{Sagan80}.


\subsection{Our Universe}

Looking out into the night sky from our vantage point on planet earth, the universe appears full of structure on many scales. Planets orbiting stars; stars bound into star clusters and galaxies; galaxies themselves organized into clusters ranging from small associations like our local group, to enormous conglomerates of many thousands of galaxies. But if we adopt a holistic mindset on the scale of around 100 \acf{Mpc}, and ignore the smaller scale density fluctuations which are so crucial to our own existence, we observe two remarkable apparent realities of our universe \citep{RydenText}:
\begin{enumerate}
\item {\bf Isotropy.} On large scales, the universe is the same in all locations. There are no special or unique locations.
\item {\bf Homogeneity.} On large scales, the universe is the same in all directions. There are no preferred directions.
\end{enumerate}
These two postulates, which are supported by observations, form the foundation of the current standard cosmological model \citep{BS01}.

A third important fact about our universe on large scales, is that it is expanding. Galaxies are moving away from all other galaxies, at a rate proportional to their distance.\footnote{Note: the linear Hubble relation only holds for relatively small cosmic distances, below a few hundred \ac{Mpc} \citep{RydenText}.} This simple relationship was discovered by \citet{Hubble29}, and can be expressed as
\begin{equation}
v \approx H_0 r,
\end{equation}
where $v$ is recessional velocity, $r$ is distance, and $H_0$ is known as the Hubble constant \citep{RydenText}. Our current best estimate is $H_0 = 67.8\pm0.77$ km/s/Mpc \citep{PlanckXVI}; this and other cosmological constants are listed in \autoref{table:constants}. The discovery of the expansion of the universe simultaneously abolished any notion that our universe was static, while also giving rise to the idea of a Big Bang origin. Extrapolating back in time, it seems that the universe was once much smaller, denser, and hotter. 

Several pieces of evidence, including the remarkable success of \acf{CMB} experiments such as the \acf{COBE} \citep{COBE96}, the \acf{WMAP} \citep{WMAP9} and Planck \citep{PlanckXVI}, provide very strong support for the Big Bang theory. First detected by \citet{PenziasWilson65}, the \ac{CMB} is an isotropic background of microwave photons that have been essentially free-streaming since the universe was a dense opaque cloud. At around 380,000 years of age, the density had dropped sufficiently, and the universe had cooled enough for neutral atoms to form, freeing the photons from constant Thomson scattering. The \ac{CMB} photons match a blackbody spectrum with temperature $2.73 K$ and have a number density of $4.11 \times 10^8 / m^3$. The tiny fluctuations in \ac{CMB} temperature on the sky, which are of order 1 part in $10^5$, are the seeds of structure formation in the universe \citep{RydenText}. Measurements of these anisotropies by \ac{WMAP} and Planck have provided cosmological parameter constraints of incredible precision (see \autoref{table:constants}).

One consequence of the expansion of the universe (and also the way it was first discovered) is that light from distant objects is redshifted as it travels to us. This shifts the spectrum of light emitted by galaxies, so that known absorption lines will appear at different wavelengths than they are observed in laboratories on earth. This cosmological redshift can be expressed in terms of the observed and the emitted wavelengths \citep{RydenText}:
\begin{equation}
z \equiv \frac{\lambda_{\rm obs} - \lambda_{\rm em}}{\lambda_{\rm em}}.
\end{equation}

Redshift is often a convenient means of indicating cosmological distances, and will be used frequently throughout this thesis. Since the universe is expanding, it is convenient to express its growth in terms of a scale factor $a(t)$. We define $a$ to be unity today ($a(t_0)=1$), and say that $a(t)<1$ in the past. In this framework we can directly convert the cosmological redshift of an object to the scale factor when its light was emitted \citep{RydenText}:
\begin{equation}
1+z = \frac{a(t_0)}{a(t_{\rm e})} = \frac{1}{a(t_{\rm e})}.
\label{eq:z}
\end{equation}

The expansion history, given by the scale factor $a(t)$, depends upon the constituents of the energy density of the universe. Numerous studies show that, in addition to obvious stuff like normal matter\footnote{Normal matter consists of atoms and other standard model particles, which, in cosmology, are typically all lumped together under the label ``baryonic matter'' despite the fact that they are not all strictly baryons in the particle physics sense \citep{RydenText}.} and radiation, the universe also contains copious amounts of cold dark matter and some form of dark energy, possibly a cosmological constant \citep{DodelsonText}. These components will be discussed in more detail below. This dark sector actually makes up the majority of the present energy density of the universe, but that was not always the case because of the way the density of each component evolves differently with the scale factor.

\subsection{Dark Matter}
\label{sec:DM}

The notion of an invisible dark matter has been around for a surprisingly long time. The idea was first proposed by \citet{Zwicky33} to explain the fact that the mass of the Coma Cluster estimated from radial velocities of member galaxies greatly exceeded the mass estimated from luminosity \citep[see also][]{Zwicky37}. Several years later a similar mismatch was observed between the rotation curves of individual galaxies and the amount of visible matter they contained \citep{Babcock39}, indicating that an enormous amount of invisible mass must extend far beyond the visible range of the galaxies. This gave rise to the concept of the dark matter halo, a much larger spherical halo of invisible dark matter existing around every galaxy and galaxy cluster \citep[see][for a review of dark matter's discovery]{Bergh99}.

Current measurements confirm the existence of dark matter, refining its abundance to around 30\% of the energy content of the entire universe -- an order of magnitude greater average density than ordinary matter. For much of the latter part of the 20th century it was thought that the dark matter could simply be normal matter that was not giving off light. Faint stars, brown dwarfs, black holes, rocky bodies, and an abundance of light weight neutrinos, were all candidates for the non-luminous material \citep{Bergh99}. The now mainstream idea of a non-baryonic cold dark matter was first introduced in 1983 by Bond et al. at the Third Moriond Astrophysics Meeting \citet{Bond83}. Cold dark matter is supported by requirements for structure formation in the universe \citep{BS01}, as well as more ``direct'' evidence of the non-interaction of dark matter studied in several merging galaxy cluster systems, including the well-known Bullet Cluster \citep{Clowe06}.

Dark matter seems to interact only through gravity, and certainly not via the electromagnetic force (it does not emit, absorb, or reflect light). The most plausible contender for the actual dark matter particle is called a \acf{WIMP}, which, as the name implies, is a massive particle that interacts only through gravity and the weak nuclear force. Common supersymmetric extensions to the Standard Model of particle physics include particles that could be candidates for \ac{WIMP}s \citep{DodelsonText}. Unfortunately, direct detection of these particles has proved extremely difficult, despite the fact that numerous groups are pursuing detection using a variety of approaches. One group has claimed a WIMP detection (this is the DAMA/LIBRA experiment which has observed an annual signal modulation), but it remains controversial and seems to be ruled out by other experiments (particularly the XENON-100 experiment) \citep{Snowmass13}.

\subsection{Dark Energy}
\label{sec:DE}

Probably the biggest mystery in all of physics is the nature of dark energy. Like dark matter, the idea has been around for many decades, but unlike dark matter it was not initially based upon observations, but rather a preconceived bias. Since this was before Hubble had discovered the expansion of space, Einstein (and many others) assumed the universe was static. But a matter dominated universe described by Einstein's theory of relativity couldn't be static -- it would contract and fall back on itself due to the gravitational potential of all the mass in it \citep{RydenText}. 

Einstein added a term to his field equations known as the cosmological constant, which was supposed to be a repulsive term that perfectly balanced the universe against gravitational collapse. These equations described how the geometry (left-hand-side) was related to the energy content (right-hand-side) of the universe:
\begin{equation}
R_{\mu\nu} - \frac{1}{2} g_{\mu\nu} R =  -\frac{8\pi G}{c^4} T_{\mu\nu} + \Lambda g_{\mu\nu}
\label{eq:Einstein}
\end{equation}
Here $R_{\mu\nu}$ and $R$ are the Ricci tensor and scalar. $G$ is Newton's gravitational constant and $c$ is the speed of light in a vacuum. The metric tensor is given by $g_{\mu\nu}$,  and the energy-momentum tensor is $T_{\mu\nu}$ \citep{Bertone05}. When the Hubble expansion was discovered, Einstein famously abandoned the cosmological constant term, $\Lambda g_{\mu\nu}$, but it has since reappeared \citep{RydenText}.

In the late 1990s, two teams of astronomers were trying to measure the deceleration of the universe, since the current expansion was expected to be slowing under the gravitational attraction of all the matter. Both the High-$z$ Supernovae Search Team \citep{Riess98} and the Supernovae Cosmology Project \citep{Perlmutter99} used type Ia supernovae as standard candles, relating the peak luminosity to the width of the light-curve, and found evidence for a non-zero cosmological constant. The universe's expansion was actually accelerating. This nobel-prize-winning discovery has been a paradigm shift for the field of cosmology.

The name dark energy refers to the unknown force causing the universe to accelerate. The simplest possibility would be a cosmological constant, perhaps a vacuum energy that is uniform throughout space.\footnote{Unfortunately, calculating the expected energy density of the vacuum from particle physics gives a value 124 orders of magnitude higher than observed. This is a major unsolved issue in theoretical physics and cosmology \citep{RydenText}.} Other more complicated theories have been invented to explain the acceleration, so dark energy is the more general term encompassing them all, but currently the data are consistent with a cosmological constant making up almost 70\% of the energy-density of the universe \citep{PlanckXVI}.

\subsection{Cosmic Dynamics}
\label{sec:Dynamics}

General Relativity stipulates that space and time are part of a single fabric, known as space-time. Events separated in this 4-dimension space-time can be described using the Robertson-Walker metric. This particular form of the metric is a direct consequence of the assumptions of homogeneity and isotropy of the universe \citep{Bertone05}:
\begin{equation}
{\rm d}s^2 = -c^2 {\rm d}t^2 +a(t)^2 \left[{\rm d}r^2 + S_{\rm k}(r)^2{\rm d}\Omega^2\right].
\label{eq:metric}
\end{equation}
Here d$t$ is an interval of proper time, d$\Omega^2 = {\rm d}\theta^2 + {\rm sin}^2\theta {\rm d}\phi^2$, and $(r,\theta,\phi)$ are the set of comoving position coordinates (comoving coordinates grow along with the Hubble expansion -- see \autoref{sec:distances}). 

The $S_{\rm k}(r)$ term is specified by the curvature of the universe, which can be {\it flat} (zero curvature, Euclidean), {\it closed} (positively curved, analogous to the surface of a sphere in 2-dimensions), or {\it open} (negatively curved, like the surface of a saddle in 2-dimensions). Explicitly,
\begin{equation}
S_{\rm k}(r) = 
    \begin{cases}
        \mathscr{R}{\rm sin}(r/\mathscr{R}), & \text{for positive curvature} \\
        r,              & \text{for zero curvature} \\
        \mathscr{R}{\rm sinh}(r/\mathscr{R}), & \text{for negative curvature.}
    \end{cases}
\end{equation}
If curvature is non-zero, then $\mathscr R$ gives the radius of curvature \citep{RydenText}. Strong limits have been placed on curvature, and it turns out our universe is flat, or extremely close to flat. The fraction of the energy density of the universe contained in curvature is less than about one part in 1000 and is consistent with zero \citep{PlanckXVI}.

Applying the Robertson-Walker metric (\autoref{eq:metric}) to the Einstein field equations (\autoref{eq:Einstein}), we obtain the Friedmann Equation:
\begin{equation}
\left( \frac{\dot a}{a} \right)^2 = \frac{8\pi G}{3} \rho_{\rm total}.
\end{equation}
Here $a=a(t)$ is the scale factor, and $\dot a$ is its first order derivative with respect to time. The total energy density of the universe $\rho_{\rm total}$ appears to be equal to the critical energy density $\rho_{\rm crit}$, which is exactly the case if the universe has zero curvature. The critical density is given by \citep{DodelsonText}:
\begin{equation}
\rho_{\rm crit} \equiv \frac{3H_0^2}{8\pi G} \approx 1.88h^2 \times 10^{-29} \frac{\rm g}{{\rm cm}^3}.
\end{equation}

Cosmologists frequently express the density of each component of the universe as a fraction of the total or critical energy density, using the notation $\Omega_i = \rho_i / \rho_{\rm crit}$. $\Omega_{\Lambda}$ represents dark energy, $\Omega_{\rm c}$ is the cold dark matter, $\Omega_{\rm b}$ is the baryonic matter, and $\Omega_{\rm r}$ is for radiation. Since the universe is flat $\sum \Omega_i = \Omega_{\Lambda} + \Omega_{\rm c} + \Omega_{\rm b} + \Omega_{\rm r} =1$. Each of these components evolves differently with the scale factor (i.e. with time, or redshift). The present values of these density parameters are given in \autoref{table:constants}, along with other cosmological constants relevant to this thesis.


%Table of cosmological constants
\begin{table*}%[b!]
 \begin{center}
    \begin{tabular}{|c|c|p{10cm}|}

      \hline
      Symbol & Value & Description \\ \hline \hline

      $\Omega_{\rm b}h^2$ & $0.02205\pm0.00028$ & Fraction of the present day energy density of the universe that is composed of baryons (times $h^2$). \\ \hline
      $\Omega_{\rm c}h^2$ & $0.1199\pm0.0027$ & Fraction of the present day energy density of the universe that is composed of cold dark matter (times $h^2$). \\ \hline
      $\tau$ & $0.089^{+0.012}_{-0.014}$ & Optical depth due to reionization. \\ \hline
      $n_{\rm s}$ & $0.9603\pm0.0073$ & Scalar spectrum power-law index. \\ \hline
      ln(10$^{10}A_{\rm s}$) & $3.089^{+0.024}_{-0.027}$ & Log power of the primordial curvature perturbations. \\ \hline 
      $100\theta_{\rm MC}$ & $1.04131\pm0.00063$ & $\theta_{\rm MC} \approx \theta_*$, the ratio of the comoving size of the sound horizon at $\tau=1$ to the angular diameter distance of the redshift at $\tau=1$. \\ \hline \hline

      $H_0$ & $67.3\pm1.2$ km/s/Mpc & Present day value of the Hubble constant, the ratio of recessional velocity to distance. \\ \hline
      $\Omega_{\rm m}$ & $0.315^{+0.016}_{-0.018}$ & Fraction of the present day energy density of the universe that is composed of pressureless matter. \\ \hline
      $\Omega_{\Lambda}$ & $0.685^{+0.018}_{-0.016}$ & Fraction of the present day energy density of the universe that is composed of dark energy. \\ \hline
      $\sigma_8$ & $0.829\pm0.012$ & Normalization of the matter power spectrum. \\ \hline
      $t_0$ & $13.817\pm0.048$ Gyr & The age of the universe. \\ \hline

    \end{tabular}
  \caption[Cosmological Constants]{Current best values of the Cosmological Constants using a combination of \ac{CMB} data from the Planck and \ac{WMAP} missions \citep{PlanckXVI}. It is quite remarkable that our entire model for the current state and evolution of the universe can be fully encapsulated by a mere 6 parameters -- the top 6 rows. The constants in the lower portion of the table are derived from these top 6 values, and are more relevant for the topics explored in this thesis. {\it Note: }The dimensionless Hubble parameter ``little $h$'' is just the Hubble Constant in units of 100 km/s/Mpc. $h \approx 0.7$ is used throughout this thesis.}
  \label{table:constants}

 \end{center}
\end{table*}


The evolution of matter (both normal matter and dark matter) is the most intuitive, as its energy density is simply inversely proportional to the volume of space, $\rho_m(t) \propto a(t)^{-3} = (1+z)^{3}$. The energy density of radiation (massless particles such as photons) scales as $\rho_r(t) \propto a(t)^{-4} = (1+z)^{4}$ because the energy of the particles drops off as the cosmological expansion increases their wavelength, yielding an extra factor of $a(t)^{-1}$ over the case for matter. Dark energy appears to be consistent with a cosmological constant, which, as the name implies, would have constant energy density $\rho_{\Lambda} \propto a(t)^{0} = (1+z)^{0}$.
%%\textcolor{red}{plot of scale factor evolution?}

It is useful to introduce the Hubble rate $H(t) \equiv {\dot a}/a$, which, at the present time $t_0$, is equal to the Hubble constant $H_0$. Thus we can re-write the Friedmann Equation, describing the evolution of the universe, in this form:
\begin{equation}
H(z)^2 = H_0^2 \left[ \Omega_{\Lambda} + \Omega_{\rm m}(1+z)^3 + \Omega_{\rm r} (1+z)^4 \right].
\end{equation}
The curvature term, which is so close to zero to be negligible, is ignored.\footnote{If included, the curvature term would scale proportionally to $(1+z)^2$.} In weak lensing, it is usually practical to ignore $\Omega_{\rm r}$ as well, and approximate the universe as being flat and composed of only matter and a cosmological constant. That will be the case in most of this thesis, and we will use the matter density $\Omega_{\rm m} \approx \Omega_{\rm c} + \Omega_{\rm b}$, absorbing the tiny fraction of the universe's baryon fraction into the cold dark matter term, and use a single term for the fractional energy density that is contributed by matter, $\Omega_{\rm m}$. 


%%\textcolor{red}{accel. eq?}
%Dod p3, 

\subsection{Distances in Cosmology}
\label{sec:distances}

Describing distances between objects in an expanding and accelerating universe is no simple task. Since physical distances are growing with the universe's expansion, a natural coordinate system to use is that of comoving coordinates, which grow along with the universe. The {\it comoving distance} between two galaxies is a constant, as long as they have no velocity relative to the Hubble expansion, while the physical distance between them grows. This physical distance is usually called the {\it proper distance}, and is simply given by
\begin{equation}
d_p(t) = a(t)r,
\end{equation}
where $r$ is the comoving radial distance, the same $r$ used in the Robertson-Walker metric (\autoref{eq:metric}) \citep{RydenText}. Since actually measuring a cosmological scale proper distance would require us to pause the universe's expansion while we extend an enormous tape measurer, we have to rely on other forms of distance, discussed below. 

Frequently we want to relate the apparent brightness of a distant astronomical object to its distance. This is especially crucial for objects of known intrinsic luminosity (often called standard candles), such as the type Ia supernovae discussed in \autoref{sec:DE}. For everyday distances here on Earth, we observe that the flux $f$ of an object with luminosity $L$ falls off as $1/($distance$)^2$. We can therefore define an analogous {\it luminosity distance} in cosmology as
\begin{equation}
d_L \equiv \sqrt{\frac{L}{4\pi f}},
\end{equation}
with the understanding that this is different than proper or comoving distance because the universe has been expanding during the time it took for the light to travel to us. In fact, this implies the relationship
\begin{equation}
d_L = (1+z)r = (1+z)d_p(t_0).
\end{equation}

Similar to the notion of a standard candle, we can imagine a standard ruler of a fixed physical length $\ell$. We can define the {\it angular diameter distance} to be the distance at which this object would have to be, in order to conform to our everyday experience of the relationship between distance, length, and angle subtended ($\theta$ [radians]):
\begin{equation}
d_A \equiv \frac{\ell}{\theta}.
\end{equation}
Here we have invoked the small angle approximation, and the angular diameter distance $d_A$ is simply related to the other distance definitions \citep{RydenText}:
\begin{equation}
d_A = \frac{r}{(1+z)}= \frac{d_L}{(1+z)^2}.
\end{equation}
Angular diameter distance is the distance measure relevant for gravitational lensing, which will be discussed in \autoref{sec:Lensing}.

%%%%%%%%%%%%%%%%%%%%%%%%%%%%%%%%%%%%%%%%%%%%%%%%%%%%%%%%%%%%%%%%%%%%%%
\section{Gravitational Lensing}
\label{sec:Lensing}

As light from distance objects in the universe makes the journey from its source to our telescopes, it is deflected by mass inhomogeneities along its path. In particular, large overdensities, such as galaxies and galaxy clusters, will cause light rays to be bent and focused, altering the images of the background objects. Einstein's theory of General Relativity predicts this effect, and specifically requires the angle of deflection to be twice that of in Newtonian gravity. During a solar eclipse in 1919, Sir Arthur Eddington measured the shifted apparent positions of stars being gravitationally lensed by the sun, providing experimental evidence for the new theory of gravity and paving the way for the future of gravitational lensing as a field \citep{BS01}.

The lensing geometry is displayed (not to scale) in \autoref{plot:lensing}. For most lensing studies the distances between astronomical objects involved is far greater than the size of the gravitational lens itself. It is therefore reasonable to approximate the path of the light ray as being bent at a sharp angle (as opposed to gradually arcing through the gravitational potential) \citep{BS01}. In analogy to refraction of light by an optical lens, this is known as the ``thin lens approximation.'' When light passes nearby an object of mass $M$, at impact parameter $b$, its path will be bent by the angle $\alpha_{\rm L}$ \citep{RydenText}:\footnote{Note that this thesis uses a similar notation to lensing reviews such as \citet{BS01} and \citet{Schneider06_WeakGravLens}, but we add the subscript ``L'' to some symbols in this chapter, pertaining to the lensing equations, to avoid confusion with the use of the same symbols in later sections of the thesis (e.g. $\beta$ will be the slope of the cluster mass-richness relation in \autoref{ch3} and \autoref{ch4}).}
\begin{equation}
\alpha_{\rm L} = \frac{4GM}{c^2 b}.
\end{equation}
This causes the background light source to appear as if it is at an angle $\theta$, when it is really at angular position $\beta_{\rm L}$, as shown in \autoref{plot:lensing}.

\begin{figure}
\begin{center}
\includegraphics[scale=0.4]{plots_intro/LensDiagram.png}
\caption[Gravitational Lensing Diagram]{Diagram showing the geometry of gravitational lensing. Light from the background source is bent by an angle $\alpha_{\rm L}$ when it passes near the gravitational lens (gray oval) on its way to the earth. While the actual source (black star) is at an angle $\beta_{\rm L}$ relative to the horizontal, its image (gray star) appears to be at an angle $\theta$. The angular diameter distances to the lens ($D_{\rm l}$), to the source ($D_{\rm s}$) and between the lens and source ($D_{\rm ls}$) are labeled.}
\label{plot:lensing}
\end{center}
\end{figure}

The fact that gravitational lensing directly probes the underlying density field along the line of sight is what makes the technique extremely valuable. All other methods for probing the matter distribution of the universe do not probe the mass itself (which is mostly dark matter), but rather the baryonic component of the mass -- stars, and interstellar gas and dust. While we expect the baryons to trace the underlying density of dark matter, there are many complicating factors (see \autoref{sec:Clusters}) that render the analogy lacking. Additionally, other means of measuring masses (such as using radial velocities) rely on assumptions about the virial equilibrium of a system which may not be satisfied.

Gravitational lensing is broadly divided into several branches depending upon the strength of the lensing effect. Strong lensing refers to the rarest and most obvious distortions, leading to images of giant arcs, Einstein rings, and multiple images of the same source. Strong lensing features are generally apparent to the eye, whereas weak lensing and microlensing are not. The first observation of gravitational lensing producing multiple images was made by \citet{Walsh79} using a lensed quasar. The first gravitationally lensed arcs were discovered nearly a decade later by \citet{Lynds86} and \citet{Soucail87,Soucail88}. Currently, over one hundred strong gravitational lenses are known \citep{Browne03,Bolton08}. Strong lensing is very useful for accurately probing the dark matter halo mass profiles of galaxies and clusters of galaxies, often yielding detailed information on lens concentration \citep{Auger10} and substructure \citep{Mao98,Dalal02}. 

Weak lensing, just as the name implies, leads to much less significant distortions. The hallmark of weak lensing is that it is a statistical effect, only measurable using ensembles of many background sources and foreground lenses. Unlike the rarity of a strong lensing event, however, weak lensing is everywhere. Essentially all light rays are distorted at least a bit while traveling to us through the inhomogeneous gravitational fields of the universe. Weak lensing itself, and different methods of measuring it, will be discussed in much more detail in \autoref{sec:Shear} and \autoref{sec:Mag} below. 

Finally, microlensing is the even weaker signature of gravitational lensing, wherein stars are lensed by low mass compact objects like black holes, brown dwarfs, and planets. Probably the most important use of microlensing has been the search for a significant \acf{MACHO} population in the Milky Way. \ac{MACHO}s were once considered a serious dark matter candidate until various microlensing experiments demonstrated that the mass density of \ac{MACHO}s was strongly insufficient to explain the missing mass in our galaxy \citep{Paczynski96,Wyrzykowski11,Sumi13}.


\subsection{Weak Lensing Shear}
\label{sec:Shear}
Weak lensing shear is the component of weak lensing that deals with shape distortion of galaxy images. If all galaxies were intrinsically circular, or of known shape, then each individual background source would provide information on the gravitational field through which its light had propagated. Instead, however, galaxies take on a variety of shapes and orientations, and their unlensed representations are impossible to know. In order to proceed, weak lensing astronomers make two critical assumptions: (1) galaxy shapes can be approximated as elliptical, and (2) the orientation of these ellipses are random in the absence of gravitational lensing \citep{BS01}. \textcolor{red}{The second point has been the subject of much study, but the affect of any intrinsic alignment of background galaxies is expected to be of order ????}

As illustrated in \autoref{plot:lensing}, the lens equation is given by $\bm{\beta}_{\rm L} = \bm{\theta} - \bm{\alpha}_{\rm L}$, where we now use bold face to indicate angular positions with two components on the sky. The deflection angle $\bm{\alpha}_{\rm L}$ can be expressed as the gradient of the lensing (or deflection) potential $\bm{\alpha}_{\rm L} = \nabla \psi$. The lensing potential $\psi(\bm{\theta})$ is the two-dimensional analogue to the Newtonian gravitational potential, and is given by \citep{Schneider06_IntroGravLensCosmology}:
\begin{equation}
\psi(\bm{\theta}) = \frac{1}{\pi} \int_{\mathbb{R}^2} \kappa(\bm{\theta'}) {\rm ln}|\bm{\theta} - \bm{\theta}'| {\rm d}^2 \theta'.
\end{equation}
Here $\kappa$ is known as the convergence, which encapsulates the magnification information to be described in \autoref{sec:Mag} below. Explicitly, the convergence is given by
\begin{equation}
\kappa(\bm{\theta}) =\frac{1}{2} \left( \frac{\partial^2 \psi(\bm{\theta})}{\partial\theta_1^2} + \frac{\partial^2 \psi(\bm{\theta})}{\partial\theta_2^2} \right)
\end{equation}
but another useful expression is the ratio
\begin{equation}
\label{eqn:kappa}
\kappa(\bm{\theta}) = \frac{\Sigma(\bm{\theta})}{\Sigma_{\rm crit}}.
\end{equation}
Here $\Sigma(\bm{\theta})$ is the two-dimensional surface mass density (with units of mass per area on the sky), and $\Sigma_{\mathrm{crit}}$ is the critical surface mass density of the lens \citep{Wright00}. The latter demarcates the separation between strong and weak gravitational lenses, depending critically on the geometry of the angular diameter distances between objects, given by
\begin{equation}
\label{eqn:sigcrit}
\Sigma_{\mathrm{crit}} = \frac{c^2}{4 \pi G} \frac{D_{\rm s}}{D_{\rm l} D_{\rm ls}}.
\end{equation}
Strong lenses (i.e. capable of forming multiple images) must have $\Sigma \ge \Sigma_{\mathrm{crit}}$ \citep{Schneider06_IntroGravLensCosmology}.

\begin{figure}
\begin{center}
\includegraphics[scale=0.3]{plots_intro/kappa_gamma.png}
\caption[$\kappa$ and $\gamma$ Diagram]{Representation demonstrating the effect of convergence $\kappa$ and shear $\gamma$ on a circular source. Diagram shows the unlensed source (green circle) and the final lensed image (black outline) for positive and negative values of both $\kappa$ and the real and imaginary components of $\gamma$. [{\it Image Credit:} TallJimbo / Wikimedia Commons / CC-BY-SA-3.0].}
\label{plot:kappagamma}
\end{center}
\end{figure}

The transformation of background objects from source (unlensed) to image (lensed) is described by the Jacobian (or amplification) matrix $\cal{A}$ \citep{DodelsonText}: 
\begin{equation}
\label{eqn:A}
{\cal A}(\bm{\theta}) = \frac{\partial \bm{\beta_{\rm L}}}{\partial \bm{\theta}}  = \left( \delta_{ij} - \frac{\partial^2 \psi(\bm{\theta})}{\partial \theta_{i} \partial \theta_j} \right) = \left( \begin{array}{cc}
{1-\kappa-\gamma_1} & {-\gamma_2} \\
{-\gamma_2} & {1-\kappa+\gamma_1} \\
\end{array} \right). 
\end{equation}
The two components of the shear $\gamma (\bm{\theta})$ can be expressed as derivates of the lensing potential:
\begin{equation}
\gamma_1 = \frac{1}{2} \left( \frac{\partial^2 \psi(\bm{\theta})}{\partial\theta_1^2} - \frac{\partial^2 \psi(\bm{\theta})}{\partial\theta_2^2} \right), 
\end{equation}
\begin{equation}
\gamma_2 = \frac{\partial^2 \psi(\bm{\theta})}{\partial\theta_1 \partial\theta_2}.
\end{equation}
The shear is often written as a complex number, $\gamma \equiv \gamma_1 + i\gamma_2 = |\gamma|{\rm e}^{2i\varphi}$. Here $|\gamma|$ and $\varphi$ indicate the amplitude and direction of distortion, respectively, which is unchanged when rotated by 180$^\circ$. 

We do not observe the true shear, but rather the reduced shear  $g(\bm{\theta}) = \gamma (\bm{\theta}) \left[1-\kappa (\bm{\theta}) \right]^{-1}$. We can thus rewrite the Jacobian matrix as:
\begin{equation}
\label{eqn:Ag}
{\cal A}(\bm{\theta}) = (1-\kappa) \left( \begin{array}{cc}
{1-g_1} & {-g_2} \\
{-g_2} & {1+g_1} \\
\end{array} \right). 
\end{equation}
In the regime of weak lensing, the convergence and shear are small, $\kappa \ll 1$ and $| \gamma| \ll 1$. Therefore it is often safe to assume that $\gamma \approx g$ \citep{Schneider06_WeakGravLens}.

The observed brightness distribution of a galaxy image $I(\bm{\theta})$ is not in general perfectly elliptical, but we can approximate it as an ellipse in the following manner. The center of the brightness distribution of the image is
\begin{equation}
\bm{\bar{\theta}} \equiv \frac{\int {\rm d}^2\theta I(\bm{\theta}) q_I( I(\bm{\theta})) \bm{\theta}}{\int {\rm d}^2\theta I(\bm{\theta}) q_I( I(\bm{\theta}))},
\end{equation}
where $q_I( I(\bm{\theta}))$ is some weight function, and we assume that the galaxy image of interest is isolated on the sky. To describe ellipticity we will be interested in the second brightness moments of the source, contained in the tensor
\begin{equation}
Q_{ij} = \frac{\int {\rm d}^2\theta I(\bm{\theta}) q_I( I(\bm{\theta})) (\theta_i - \bar{\theta_i})(\theta_j - \bar{\theta_j})}{\int {\rm d}^2\theta I(\bm{\theta}) q_I( I(\bm{\theta}))},
\end{equation}
where $i,j \in (1,2)$ \citep{Schneider06_WeakGravLens}. The size $\omega$ of the galaxy image is just a function of the diagonal components of this matrix,
\begin{equation} 
\omega = (Q_{11}Q_{22} - Q_{12}^2)^{1/2},
\end{equation}
while the shape, or ellipticity, of the image involves the off-diagonal elements:
\begin{equation} 
\chi \equiv \frac{Q_{11} - Q_{22} + 2i Q_{12}}{Q_{11} + Q_{22}} ,
\end{equation}
\begin{equation} 
\epsilon \equiv \frac{Q_{11} - Q_{22} + 2i Q_{12}}{Q_{11} + Q_{22} + 2(Q_{11}Q_{22} - Q_{12}^2)^{1/2}}.
\end{equation}
The complex ellipticity can be characterized by either of $\chi$ or $\epsilon$, which are simply related and interchangeable (in different situations one may be easier to work with). In the weak lensing regime $\gamma \approx g \approx \langle \epsilon \rangle \approx  \langle \chi \rangle /2$ \citep{BS01}. 

In analogy with the image center $\bm{\bar{\theta}}$ and tensor of second brightness moments $Q_{ij}$, one can define the same quantities for the unlensed source center $\bm{\bar{\beta}}$ and tensor of second brightness moments $Q_{ij}^{(s)}$. The relation between the source and image tensors is
\begin{equation}
Q^{(s)} = {\cal A} Q {\cal A}^{\rm T} = {\cal A} Q {\cal A},
\end{equation}
where ${\cal A} \equiv {\cal A}(\bm{\theta})$ is the Jacobian defined in \autoref{eqn:A} and \autoref{eqn:Ag}. Further calculation shows that the complex image ellipticities ($\epsilon$ or $\chi$) can be related to the source ellipticities ($\epsilon^{(s)}$ or $\chi^{(s)}$) through the reduced shear \citep{BS01}.

In this thesis we are concerned with a particular manifestation of gravitational lensing -- lensing by galaxy clusters. If we consider an circularly symmetric mass density on the sky (an idealized galaxy cluster), then we expect the shear distortion to be oriented tangential to the center of the lens. It is therefore useful in cluster lensing (and also in galaxy-galaxy lensing) to express the shear in terms of tangential and rotated (or cross) components:
\begin{equation}
\gamma_{\rm t} = -{\rm Re}\left[ \gamma {\rm e}^{-2i\phi} \right],  
\end{equation}
\begin{equation}
\gamma_{\rm r} = -{\rm Im}\left[ \gamma {\rm e}^{-2i\phi} \right].
\end{equation}
Here the angle $\phi$ is the azimuthal angle measured about the center of the lens \citep{Schneider06_WeakGravLens}. The rotated shear (which would represent a curl component) should be consistent with zero, and is often used as a check of systematic effects. Even though any single galaxy cluster (or other lens) is likely not perfectly azimuthally symmetric, we expect a stack of many galaxy clusters to yield a symmetric profile on average.

Similar to the surface mass density representation of the convergence (\autoref{eqn:kappa}), we can then relate the tangential shear to the differential surface mass density of the lens:
\begin{equation}
\gamma_{\rm t}(\theta) = \frac{\Delta\Sigma(\theta)}{\Sigma_{\rm crit}},
\end{equation}
where $\theta$ now specifies the radial angle of separation between the lens center and the source image. The differential surface mass density is defined to equal the difference between the average surface mass density interior to $\theta$ and the surface mass density at $\theta$ \citep{Wright00}: 
\begin{equation}
\Delta\Sigma(\theta) \equiv \overline{\Delta\Sigma}(< \theta) - \Delta\Sigma(\theta).
\end{equation}
Given an expression for the mass density profile of a gravitational lens, and angular diameter distances involved, the expected tangential shear profile can be derived. Useful models for galaxy cluster masses, and the resulting $\gamma_{\rm t}(\theta)$ and $\Delta\Sigma(\theta)$ profiles, will be given in \autoref{sec:Clusters}.

\subsection{Weak Lensing Magnification}
\label{sec:Mag}
Weak lensing magnification is, to first order, a measure of the convergence of a lensing mass.  It can be detected through the stretching of solid angle on the sky, which leads to the amplification of source flux, since lensing conserves surface brightness. In general, two different approaches can be taken to measure magnification.  The method we employ involves observing the effects on source number densities that result from source amplification in a flux-limited survey. An alternative approach is to observe coherent variations in source sizes and flux, but this method suffers from many of the limitations facing shear analysis.

Magnification affects the source number densities in two ways, and the one that dominates is determined by the intrinsic magnitude number counts of the sources in question.  Simply put, the brightest sources, which usually have steep number counts, will exhibit an {\it increase} in number density when lensed, as the amplification allows more objects to be detected, while the number density of the faintest sources, having relatively shallow number counts, will {\it decrease} \citep{Narayan89}.

Compared to shear measurements, magnification exhibits a slightly lower signal-to-noise ($S/N$) ratio, the reason it has been largely ignored until recently.  However, what magnification lacks in signal strength, it makes up for in terms of its ability to be applied to lenses at higher redshift and to poorly resolved sources \citep{Waerbeke10}.  Since shear studies require measurements of galaxy shapes, in order for a source to be used it must necessarily be well resolved.  This is in stark contrast to magnification studies using source number densities, which have no such requirement for the sources to be resolved at all.  In principle only source magnitudes, redshifts, and positions relative to a lens must be known.  This simple fact makes it possible to extend weak lensing magnification analyses to a much higher redshift than possible for shear, and allows a much higher source density to be included in the analysis.  

Because the constraint on source resolution is considerably relaxed, ground-based observations can be incorporated to a greater degree with magnification than for shear, as we are not concerned with correcting for the smearing of the image due to the Point Spread Function. From the financial perspective, magnification studies are therefore extremely cheap to carry out, since the excellent resolution of space-based telescopes is not required. See \citet{Waerbeke10}, \citet{RozoSchmidt10}, and \citet{Umetsu11}, for more detailed discussions of the benefits of combining magnification with shear in gravitational lensing studies.

The magnification factor $\mu$ is the inverse determinant of the amplification matrix, which maps the image deformation from the source to observer frame, and describes the first order effects of gravitational lensing \citep{BS01}.
\begin{equation}
\mu = \frac{1}{\mathrm{det} \cal{A}} = 
\frac{1}{(1-\kappa)^2 - \left|\gamma\right|^2} \approx 1+2\kappa
\label{mu}
\end{equation}
The magnification is simply a measure of the convergence ($\kappa$) of light rays due to the projected mass along the line-of-sight.  Assuming a model for a density profile, e.g. SIS or NFW, $\mu$ can then directly yield the surface-mass-density of the lens.

The number of observed source galaxies, as a function of magnitude and redshift, $n(m,z)$, is related to the intrinsic number $n_0(m,z)$ that would be observed in the absence of lensing through an equation involving properties of both the lens and source:
\begin{equation}
n(m,z)dm = \mu ^{\alpha -1} n_0(m,z)dm.
\end{equation}
where $\alpha$ is a property of the background sources, defined according to
\begin{equation}
\alpha \equiv \alpha(m,z) = 2.5 \frac{\mathrm d}{\mathrm d \it m} \log n_0(m,z).
\label{alpha}
\end{equation}
The form of this relationship was first demonstrated by \citet{Narayan89}, as it applied to lensed quasar number densities, but can be easily generalized to any galaxy type as long as one has a means of obtaining the slope of the number counts $\alpha$.

This shows that distant source galaxies, lensed by an intervening concentration of mass, may have their observed number counts increased {\it or} decreased depending on the sign of the quantity $\alpha -1$.  Sources for which $\alpha -1 > 0$ will appear to be correlated on the sky with a lens position, while sources with $\alpha -1 < 0$ will be anti-correlated, as a dearth of objects will be observed in the vicinity of a lens.  The number density of galaxies for which the intrinsic number count slope gives $\alpha -1 \approx 0$ will essentially be unaffected by lensing magnification, as the dilution and amplification effects will cancel, and no correlation signal will be observed for these objects \citep{Scranton05}.

The first measurement? or calculation? was by \citet{Broadhurst95}
Important early lensing calculations by \citet{KS93} First description of lensing for measuring mass was by \citet{Tyson84}. first realization that quasar excess around galaxies was due to lensing was by \citet{Narayan89}.

\subsection{Magnification vs. Shear}

%%%%%%%%%%%%%%%%%%%%%%%%%%%%%%%%%%%%%%%%%%%%%%%%%%%%%%%%%%%%%%%%%%%%%
\section{Galaxy Clusters}
\label{sec:Clusters}

\subsection{Clusters as Cosmological Probes}

\subsection{Intracluster Physics}

%%%%%%%%%%%%%%%%%%%%%%%%%%%%%%%%%%%%%%%%%%%%%%%%%%%%%%%%%%%%%%%%%%%%%%
\section{Impact of this Thesis}
\label{sec:Impact}

This work contained in this thesis has pushed the boundaries of what knowledge can be extracted from weak lensing surveys. By including intrinsically smaller and fainter background sources, which cannot be used in conventional weak lensing studies, we pave the way for a more optimal use of survey data. These gravitationally-lensed sources, which are too small for reliable shape or size measurements, can still be included in a lensing analysis by using the flux magnification formalism described in \autoref{sec:Mag}. The author of this thesis has proven the utility of measuring magnification, through several key publications which appear as chapters in this work, and carried out thorough studies of the systematic effects which provide limitations. 

Prior to this thesis research, weak lensing was dominated by the shear method. This was originally motivated by some early work showing that the signal-to-noise for shear was several times larger than for magnification \citep{Schneider00}. While it is true that, for a fixed sample of galaxies, there is less scatter in galaxy shapes than in galaxy positions, the latter is far easier to measure. This simple fact has motivated the research herein. As lensing studies push to higher redshift, and increasingly rely on blurry ground-based data, we have elevated confidence in our measurements of source positions over difficult shape determinations. 

When this thesis work began, only a handful of magnification studies had been completed. The first ground-breaking theoretical formulation of how number densities of sources could be used to measure masses of clusters was laid out in 1995 by \citet{Broadhurst95}, but it took another 20 years before the first convincing observational detection was made \citep{Scranton05}. Following this significant $8\sigma$ detection of galaxy-magnified quasars in the \acf{SDSS}, several studies followed, achieving magnification detections for lensing of normal galaxies \citep{Hildebrandt09b}, and of blue galaxies behind strong lensing clusters \citep{Umetsu11}. These studies all stopped short of deriving scientifically useful results from the magnification measurements -- they either represented proof-of-concept studies for a new technique, or they demonstrated consistency with a lensing interpretation of the signal.

The work in this thesis made major steps forward in the area of lensing magnification. The author performed the first-ever measurement of magnification by stacked galaxy groups in 2012 (See \autoref{ch2}). This particular work was also the first time that shear and magnification mass estimates and signal-to-noise had been compared \citep{Ford12}. Following this influential work in the \acf{COSMOS} survey, the thesis author transitioned focus to the much larger astronomical survey known as the \acf{CFHTLenS}. 

Two important studies resulted from magnification analyses of the \ac{CFHTLenS} for this thesis (See \autoref{ch3} and \autoref{ch4}). First of all, the most significant magnification detection thus far (at $9.7\sigma$) was published in \citet{Ford14}. More importantly, however, that work moved beyond simple magnification-detection to actual science. The thesis author measured masses of stacked galaxy clusters binned as a function of different attributes (redshift and richness), and the dependence of a magnification signal on these parameters was seen for the first time. A mass-richness scaling relation was determined solely from the magnification results, which is a useful tool for making cosmological inferences from optical cluster surveys, as discussed in \autoref{sec:Clusters}. 

This work contained the important inclusion of a means of accounting for one of the dominant systematic effects for magnification, the contamination of the background sources with low-redshift objects. The formalism was extended from earlier work by \citet{Hildebrandt13}, but allowing for different contamination fractions and models for the halo occupation distribution of the galaxy contaminants. This was the first time that galaxy cluster lenses could be used for magnification in a redshift range where there was known source contamination. Prior to this work, the redshift ranges of overlap had to be avoided because the physically-induced cross-correlations of lens and source objects overwhelmed, and could not be separated from, the magnification signal.

Arguably the most important magnification result in all the literature to date is contained in \autoref{ch4} of this thesis. After the semi-blind magnification analysis of \citet{Ford14}, \citet{Ford15} followed suit with an identical treatment of the same cluster sample, but this time using the weak lensing shear approach. This study contained a detailed comparison between cluster masses measured with the two independent techniques, as a function of different cluster attributes, and contained valuable insights regarding systematic effects that are still important to resolve for magnification. Moving forward, this work frames the case for including magnification, and also pin-points some important issues that must be addressed in future work (see \autoref{ch:conc}).

%%%%%%%%%%%%%%%%%%%%%%%%%%%%%%%%%%%%%%%%%%%%%%%%%%%%%%%%%%%%%%%%%%%%%%
\section{Thesis Overview}
\label{sec:Overview}

The body of this thesis is composed of three published studies that develop the weak gravitational lensing magnification technique, particularly for the study of galaxy clusters, and compare with results using the complementary and much more ubiquitous weak lensing shear approach:
\begin{itemize}
\item \autoref{ch2} contains the first magnification study of galaxy groups and first comparison with shear for stacked lens samples. The data are X-ray selected groups, and high-redshift Lyman-break galaxies in the ac{COSMOS} field.
\item \autoref{ch3} represents the highest-significance magnification detection, and the first magnification study that could be binned as a function of cluster parameters. The data are optically-selected galaxy clusters and high-redshift Lyman-break galaxies in the ac{CFHTLenS} field.
\item \autoref{ch4} is the follow-up shear analysis of the same cluster sample presented in the previous chapter, using the ac{CFHTLenS} shear catalog for background source shape measurements.
\end{itemize}
Finally, \autoref{ch:conc} wraps up with conclusions on the topic of cluster studies using both magnification and shear, and briefly outlines future directions for progress within the field.

%%%%%%%%%%%%%%%%%%%%%%%%%%%%%%%%%%%%%%%%%%%%%%%%%%%%%%%%%%%%%%%%%%%%%%
\endinput
Any text after an \endinput is ignored.
You could put scraps here or things in progress.
